\subsection{Package \lstinline!cryptocast.crypto!}

\subsubsection{Class \lstinline|LagrangeInterpolation<T>|}
Performs a lagrange interpolation of a polynomial \\
\begin{tikzpicture}
\umlclass[]{LagrangeInterpolation<T>}{

}{
+ computeCoefficients() : T[]
}
\end{tikzpicture}



\begin{itemize}
\item \lstinline|<T>|: The type of items of the polynomial's field
\end{itemize}


\textbf{Constructors}
\begin{itemize}
\item \lstinline|public| \lstinline|LagrangeInterpolation|\lstinline|(Polynomial<T> poly)|\\
Initializes the algorithm
\begin{itemize}
\item \lstinline|poly|: The polynomial to interpolate
\end{itemize}



\end{itemize}


\textbf{Methods}
\begin{itemize}
\item \lstinline|public T[]| \lstinline|computeCoefficients|\lstinline|()|\\
Returns: The lagrange coefficients of the associated polynomial



\end{itemize}

\subsubsection{Class \lstinline|IntegersModuloPrime|}
The field $\mathbb{Z}/p\mathbb{Z}$ of integers modulo a prime $p$ \\
\begin{tikzpicture}
\umlclass[]{IntegersModuloPrime}{

}{
+ add() : BigInteger \\ + multiply() : BigInteger \\ + negate() : BigInteger \\ + invert() : BigInteger \\ + zero() : BigInteger \\ + one() : BigInteger \\ + randomElement() : BigInteger
}
\end{tikzpicture}



\textbf{Superclasses and Interfaces}
\begin{itemize}
\item \lstinline|cryptocast.crypto.Field<BigInteger>|
\end{itemize}



\textbf{Constructors}
\begin{itemize}
\item \lstinline|public| \lstinline|IntegersModuloPrime|\lstinline|(BigInteger p)|\\
Initializes the field
\begin{itemize}
\item \lstinline|p|: A prime number
\end{itemize}



\end{itemize}


\textbf{Methods}
\begin{itemize}
\item \lstinline|public BigInteger| \lstinline|add|\lstinline|(BigInteger a, BigInteger b)|




\item \lstinline|public BigInteger| \lstinline|multiply|\lstinline|(BigInteger a, BigInteger b)|




\item \lstinline|public BigInteger| \lstinline|negate|\lstinline|(BigInteger a)|




\item \lstinline|public BigInteger| \lstinline|invert|\lstinline|(BigInteger a)|




\item \lstinline|public BigInteger| \lstinline|zero|\lstinline|()|




\item \lstinline|public BigInteger| \lstinline|one|\lstinline|()|




\item \lstinline|public BigInteger| \lstinline|randomElement|\lstinline|()|




\end{itemize}

\subsubsection{Class \lstinline|Polynomial<T>|}
A polynomial $P$ over a field \\
\begin{tikzpicture}
\umlclass[]{Polynomial<T>}{

}{
+ getField() : Field<T> \\ + evaluate() : T \\ + evaluateMulti() : T[] \\ + getCoefficient() : T \\ + getDegree() : int \\ \umlstatic{+ createRandomPolynomial() : Polynomial<T>}
}
\end{tikzpicture}



\begin{itemize}
\item \lstinline|<T>|: The type of the field's elements
\end{itemize}


\textbf{Constructors}
\begin{itemize}
\item \lstinline|public| \lstinline|Polynomial|\lstinline|(Field<T> field, T[] coefficients)|\\
Initializes a polynomial
\begin{itemize}
\item \lstinline|field|: An instance of the field over which the polynomial is formed
\item \lstinline|coefficients|: The coefficients $c_i$ of the polynomial ($0 \leq i \leq n$).
 The polynomial is defined as $P(x) := \sum_{i=0}^n c_i x^i = c_0 + c_1
 x + ... + c_n x^n$
\end{itemize}



\end{itemize}


\textbf{Methods}
\begin{itemize}
\item \lstinline|public Field<T>| \lstinline|getField|\lstinline|()|\\
Returns: The field associated with this polynomial



\item \lstinline|public T| \lstinline|evaluate|\lstinline|(T x)|\\
Evaluates the polynomial at a single point x.
\begin{itemize}
\item \lstinline|x|: The
\end{itemize}

Returns: P(x)

\item \lstinline|public T[]| \lstinline|evaluateMulti|\lstinline|(T[] xs)|\\
Evaluates the polynomial at multiple points in time complexity $\Theta(n\cdot\log
 n)$ where $n$ is the degree of the polynomial
\begin{itemize}
\item \lstinline|xs|: The points $x_i$ to evaluate
\end{itemize}

Returns: The array a defined by $a_i := P(x_i)$

\item \lstinline|public T| \lstinline|getCoefficient|\lstinline|(int i)|\\
Returns: $c_i$
\begin{itemize}
\item \lstinline|i|: The index of the coefficient to get ($0 \leq i \leq n$), where
          $n$ is the degree of the polynomial
\end{itemize}



\item \lstinline|public int| \lstinline|getDegree|\lstinline|()|\\
Returns: The degree of the polynomial



\item \lstinline|public static Polynomial<T>| \lstinline|createRandomPolynomial|\lstinline|(Field<T> field, int degree)|\\
Generates a random polynomial over the field
\begin{itemize}
\item \lstinline|field|: An instance of the field over which the polynomial is formed
\item \lstinline|degree|: The degree of the generated polynomial
\end{itemize}

Returns: The generated polynomial

\end{itemize}

\subsubsection{Class \lstinline|Field<T>|}
Represents a field over values of type T \\
\begin{tikzpicture}
\umlclass[type=abstract]{Field<T>}{

}{
\umlvirt{+ add() : T} \\ \umlvirt{+ multiply() : T} \\ \umlvirt{+ negate() : T} \\ \umlvirt{+ invert() : T} \\ \umlvirt{+ zero() : T} \\ \umlvirt{+ one() : T} \\ \umlvirt{+ randomElement() : T} \\ \umlvirt{+ subtract() : T} \\ \umlvirt{+ divide() : T} \\ \umlvirt{+ pow() : T}
}
\end{tikzpicture}



\begin{itemize}
\item \lstinline|<T>|: The values we work on
\end{itemize}


\textbf{Constructors}
\begin{itemize}
\item \lstinline|public| \lstinline|Field|\lstinline|()|




\end{itemize}


\textbf{Methods}
\begin{itemize}
\item \lstinline|public abstract T| \lstinline|add|\lstinline|(T a, T b)|\\
Adds two elements of the field
\begin{itemize}
\item \lstinline|a|: first element
\item \lstinline|b|: second element
\end{itemize}

Returns: The value $a + b$

\item \lstinline|public abstract T| \lstinline|multiply|\lstinline|(T a, T b)|\\
Multiplies two elements of the field
\begin{itemize}
\item \lstinline|a|: first element
\item \lstinline|b|: second element
\end{itemize}

Returns: The value $a \cdot b$

\item \lstinline|public abstract T| \lstinline|negate|\lstinline|(T a)|\\
Returns: The additive inverse $-a$ of $a$
\begin{itemize}
\item \lstinline|a|: An element of the field
\end{itemize}



\item \lstinline|public abstract T| \lstinline|invert|\lstinline|(T a)|\\
Returns: The multiplicative inverse $a^{-1}$ of $a$
\begin{itemize}
\item \lstinline|a|: An element of the field
\end{itemize}



\item \lstinline|public abstract T| \lstinline|zero|\lstinline|()|\\
Returns: The zero element of the field



\item \lstinline|public abstract T| \lstinline|one|\lstinline|()|\\
Returns: The one element of the field



\item \lstinline|public abstract T| \lstinline|randomElement|\lstinline|()|\\
Returns: A random element of the field



\item \lstinline|public T| \lstinline|subtract|\lstinline|(T a, T b)|\\
Subtracts two elements of the field
\begin{itemize}
\item \lstinline|a|: first element
\item \lstinline|b|: second element
\end{itemize}

Returns: The value $a - b$

\item \lstinline|public T| \lstinline|divide|\lstinline|(T a, T b)|\\
Divides two elements of the field
\begin{itemize}
\item \lstinline|a|: first element
\item \lstinline|b|: second element
\end{itemize}

Returns: The value $\frac{a}{b}$

\item \lstinline|public T| \lstinline|pow|\lstinline|(T a, int e)|\\
Raises an element of the field to an integer power
\begin{itemize}
\item \lstinline|a|: The element of the field
\item \lstinline|e|: The exponent
\end{itemize}

Returns: The value $a^e$

\end{itemize}

\subsubsection{Class \lstinline|BroadcastEncryptionServer<ID>|}
The server side of a broadcast encryption scheme. \\
\begin{tikzpicture}
\umlclass[]{BroadcastEncryptionServer<ID>}{

}{
+ run() \\ + send() \\ + revoke()
}
\end{tikzpicture}



\textbf{Superclasses and Interfaces}
\begin{itemize}
\item \lstinline|cryptocast.comm.OutChannel|
\item \lstinline|java.lang.Runnable|
\end{itemize}

\begin{itemize}
\item \lstinline|<ID>|: The type of the identities
\end{itemize}


\textbf{Constructors}
\begin{itemize}
\item \lstinline|public| \lstinline|BroadcastEncryptionServer|\lstinline|(MessageOutChannel inner, BroadcastSchemeUserManager<ID> context, Encryptor<BigInteger> enc)|\\
Initializes a broadcast encryption server.
\begin{itemize}
\item \lstinline|inner|: The message-based communication channel to send outgoing data to
\item \lstinline|context|: The user management context
\item \lstinline|enc|: The encryption context
\end{itemize}



\end{itemize}


\textbf{Methods}
\begin{itemize}
\item \lstinline|public void| \lstinline|run|\lstinline|()|\\
Run the worker that handles periodic group key broadcasts and sends
 queued data packages.



\item \lstinline|public void| \lstinline|send|\lstinline|(byte[] data)|\\
Send plaintext data to the channel. It will be encryted and broadcasted
 on the fly.
\begin{itemize}
\item \lstinline|data|: The data to send
\end{itemize}



\item \lstinline|public void| \lstinline|revoke|\lstinline|(ID id)|\\
Revoke a user.
\begin{itemize}
\item \lstinline|id|: The identity of the user
\end{itemize}



\end{itemize}

\subsubsection{Class \lstinline|BroadcastEncryptionClient|}
The client side of a broadcast encryption scheme. \\
\begin{tikzpicture}
\umlclass[]{BroadcastEncryptionClient}{

}{
+ recv()
}
\end{tikzpicture}



\textbf{Superclasses and Interfaces}
\begin{itemize}
\item \lstinline|cryptocast.comm.InChannel|
\end{itemize}



\textbf{Constructors}
\begin{itemize}
\item \lstinline|public| \lstinline|BroadcastEncryptionClient|\lstinline|(MessageInChannel inner, Decryptor<BigInteger> dec)|\\
Initializes a broadcast encryption client.
\begin{itemize}
\item \lstinline|inner|: The message-based underlying communication channel.
\item \lstinline|dec|: The decryption context
\end{itemize}



\end{itemize}


\textbf{Methods}
\begin{itemize}
\item \lstinline|public void| \lstinline|recv|\lstinline|(int size, byte[] buffer)|\\
Receive data from the channel. It is decrypted on the fly.
\begin{itemize}
\item \lstinline|size|: amount of bytes to receive
\item \lstinline|buffer|: the target buffer
\end{itemize}



\end{itemize}

\subsubsection{Interface \lstinline|Decryptor<S>|}
A strategy to decrypt a single secret \\
\begin{tikzpicture}
\umlclass[type=abstract]{Decryptor<S>}{

}{
\umlvirt{+ decrypt() : S}
}
\end{tikzpicture}



\begin{itemize}
\item \lstinline|<S>|: the type of the secret
\end{itemize}



\textbf{Methods}
\begin{itemize}
\item \lstinline|public S| \lstinline|decrypt|\lstinline|(byte[] cipher)|\\
Decrypts a secret.
\begin{itemize}
\item \lstinline|cipher|: The encrypted secret
\end{itemize}

Returns: The decrypted secret

\end{itemize}

\subsubsection{Interface \lstinline|BroadcastSchemeUserManager<ID>|}
Manages a set of user identites \\
\begin{tikzpicture}
\umlclass[type=abstract]{BroadcastSchemeUserManager<ID>}{

}{
\umlvirt{+ getIdentity() : ID} \\ \umlvirt{+ revoke()} \\ \umlvirt{+ isRevoked() : boolean}
}
\end{tikzpicture}



\begin{itemize}
\item \lstinline|<ID>|: The type of the identities
\end{itemize}



\textbf{Methods}
\begin{itemize}
\item \lstinline|public ID| \lstinline|getIdentity|\lstinline|(int i)|\\
Returns: The identity with the given index
\begin{itemize}
\item \lstinline|i|: An index
\end{itemize}



\item \lstinline|public void| \lstinline|revoke|\lstinline|(ID id)|\\
Revokes a user
\begin{itemize}
\item \lstinline|id|: The identity of the user
\end{itemize}



\item \lstinline|public boolean| \lstinline|isRevoked|\lstinline|(ID id)|\\
Returns: whether the user is revoked
\begin{itemize}
\item \lstinline|id|: The identity of the user
\end{itemize}



\end{itemize}

\subsubsection{Interface \lstinline|Encryptor<S>|}
A strategy to encrypt a single secret \\
\begin{tikzpicture}
\umlclass[type=abstract]{Encryptor<S>}{

}{
\umlvirt{+ encrypt() : byte[]}
}
\end{tikzpicture}



\begin{itemize}
\item \lstinline|<S>|: the type of the secret
\end{itemize}



\textbf{Methods}
\begin{itemize}
\item \lstinline|public byte[]| \lstinline|encrypt|\lstinline|(S secret)|\\
Encrypts a secret
\begin{itemize}
\item \lstinline|secret|: the secret
\end{itemize}

Returns: The cipher text

\end{itemize}

\subsubsection{Interface \lstinline|BroadcastSchemeKeyManager<ID>|}
Manages the private keys of a set of users. \\
\begin{tikzpicture}
\umlclass[type=abstract]{BroadcastSchemeKeyManager<ID>}{

}{
\umlvirt{+ getPersonalKey() : PrivateKey}
}
\end{tikzpicture}



\begin{itemize}
\item \lstinline|<ID>|: The type of the user identities
\end{itemize}



\textbf{Methods}
\begin{itemize}
\item \lstinline|public PrivateKey| \lstinline|getPersonalKey|\lstinline|(ID id)|\\
Returns: The private key of the user
\begin{itemize}
\item \lstinline|id|: The identity to look up
\end{itemize}



\end{itemize}

\subsubsection{Interface \lstinline|ShareCombinator<S, T>|}
Implements a strategy to restore a secret from a number of shares. \\
\begin{tikzpicture}
\umlclass[type=abstract]{ShareCombinator<S, T>}{

}{
\umlvirt{+ restore() : Optional<S>}
}
\end{tikzpicture}



\begin{itemize}
\item \lstinline|<S>|: The type of the secret
\item \lstinline|<T>|: The type of the shares
\end{itemize}



\textbf{Methods}
\begin{itemize}
\item \lstinline|public Optional<S>| \lstinline|restore|\lstinline|(Collection<T> shares)|\\
Restores a secret from several shares.

Returns: The reconstructed secret or absent if the information represented
 by the given shares is insufficient to restore it.

\end{itemize}


