\subsection{Package \lstinline!cryptocast.util!}
Utility classes that are not specific to CryptoCast and don't fit any of the other packages.
\subsubsection{Class \lstinline|CommandLineInterface|}
A simple framework for command line programs. \\
\begin{tikzpicture}
\umlclass[type=abstract]{CommandLineInterface}{

}{
+ run() : int \\ \# start() : void \\ \# printf() : void \\ \# print() : void \\ \# println() : void \\ \# getErrorFormat() : String \\ \# fatalError() : void \\ \# exit() : void \\ \# usage() : void \\ \# getBasicUsage() : String \\ \# printAdditionalUsage() : void
}
\end{tikzpicture}





\textbf{Constructors}
\begin{itemize}
\item \lstinline|public| \lstinline|CommandLineInterface|\lstinline|(InputStream in, PrintStream out, PrintStream err)|\\
Initializes a new CLI instance
\begin{itemize}
\item \lstinline|in|: The stream for program input
\item \lstinline|out|: The stream for program output
\item \lstinline|err|: The stream for error output
\end{itemize}



\end{itemize}


\textbf{Methods}
\begin{itemize}
\item \lstinline|public int| \lstinline|run|\lstinline|(String[] args)|\\
Runs the application.
\begin{itemize}
\item \lstinline|args|: The command line arguments
\end{itemize}

Returns: The exit code

\item \lstinline|protected abstract void| \lstinline|start|\lstinline|(String[] args)|\\
The main program logic (must be overridden by subclasses)
\begin{itemize}
\item \lstinline|args|: The command line arguments
\end{itemize}



\item \lstinline|protected void| \lstinline|printf|\lstinline|(String format, Object[] args...)|\\
Prints a string to the output stream
\begin{itemize}
\item \lstinline|format|: The string to print (printf format string)
\item \lstinline|args|: The printf arguments
\end{itemize}



\item \lstinline|protected void| \lstinline|print|\lstinline|(String str)|\\
Prints a string to the output stream
\begin{itemize}
\item \lstinline|str|: The string to print
\end{itemize}



\item \lstinline|protected void| \lstinline|println|\lstinline|(String str)|\\
Prints a string to the output stream after appending a newline
\begin{itemize}
\item \lstinline|str|: The string to print
\end{itemize}



\item \lstinline|protected String| \lstinline|getErrorFormat|\lstinline|()|\\
Returns: the string format to use for writing error messages to the
 screen.



\item \lstinline|protected void| \lstinline|fatalError|\lstinline|(String format, Object[] args...)|\\
Prints an error and exits
\begin{itemize}
\item \lstinline|format|: The error message (printf format string)
\item \lstinline|args|: The printf arguments
\end{itemize}



\item \lstinline|protected void| \lstinline|exit|\lstinline|(int status)|\\
Exits the application
\begin{itemize}
\item \lstinline|status|: The exit code
\end{itemize}



\item \lstinline|protected void| \lstinline|usage|\lstinline|()|\\
Prints usage information



\item \lstinline|protected String| \lstinline|getBasicUsage|\lstinline|()|\\
Returns: basic usage information for the program (should be overridden)



\item \lstinline|protected void| \lstinline|printAdditionalUsage|\lstinline|()|\\
Prints additional usage information (may be overridden)



\end{itemize}

\subsubsection{Class \lstinline|CommandLineInterface.Exit|}
Signals the exit of the application. \\
\begin{tikzpicture}
\umlclass[]{CommandLineInterface.Exit}{

}{
+ getStatus() : int
}
\end{tikzpicture}



\textbf{Superclasses and Interfaces}
\begin{itemize}
\item \lstinline|java.lang.Throwable|
\end{itemize}



\textbf{Constructors}
\begin{itemize}
\item \lstinline|public| \lstinline|CommandLineInterface.Exit|\lstinline|(int status)|\\
Initializes a new Exit instance
\begin{itemize}
\item \lstinline|status|: The exit code
\end{itemize}



\end{itemize}


\textbf{Methods}
\begin{itemize}
\item \lstinline|public int| \lstinline|getStatus|\lstinline|()|\\
Returns: the exit code



\end{itemize}

\subsubsection{Class \lstinline|InteractiveCommandLineInterface|}
A framework class to implement interactive command-line interfaces. The class implements
 a read-parse-execute main loop and provides hooks for subclasses to implement the missing
 functionality. \\
\begin{tikzpicture}
\umlclass[type=abstract]{InteractiveCommandLineInterface}{

}{
\# start() : void \\ \# mainloop() : void \\ \# performCommand() : void \\ \# error() : void \\ \# getPrompt() : String
}
\end{tikzpicture}



\textbf{Superclasses and Interfaces}
\begin{itemize}
\item \lstinline|cryptocast.util.CommandLineInterface|
\end{itemize}



\textbf{Constructors}
\begin{itemize}
\item \lstinline|public| \lstinline|InteractiveCommandLineInterface|\lstinline|(InputStream in, PrintStream out, PrintStream err)|\\
Initializes a new interactive CLI instance
\begin{itemize}
\item \lstinline|in|: The stream for program input
\item \lstinline|out|: The stream for program output
\item \lstinline|err|: The stream for error output
\end{itemize}



\end{itemize}


\textbf{Methods}
\begin{itemize}
\item \lstinline|protected void| \lstinline|start|\lstinline|(String[] args)|\\
The main program logic. This method just starts the main loop.
\begin{itemize}
\item \lstinline|args|: The command line arguments (ignored by default)
\end{itemize}



\item \lstinline|protected void| \lstinline|mainloop|\lstinline|()|\\
Starts the interactive Prompt-Read-Evaluate main loop.



\item \lstinline|protected abstract void| \lstinline|performCommand|\lstinline|(String cmd, String[] args)|\\
Executes the given command with the given arguments. Must be implemented by subclasses.
\begin{itemize}
\item \lstinline|cmd|: The command name
\item \lstinline|args|: The command arguments
\end{itemize}



\item \lstinline|protected void| \lstinline|error|\lstinline|(String format, Object[] args...)|\\
Helper function to trigger an error withing a command's execution and break out to
 the main loop.
\begin{itemize}
\item \lstinline|format|: The format string
\item \lstinline|args|: The format string arguments
\end{itemize}



\item \lstinline|protected String| \lstinline|getPrompt|\lstinline|()|\\
Returns: The input prompt



\end{itemize}

\subsubsection{Class \lstinline|InteractiveCommandLineInterface.CommandError|}
An error within one of the commands. Will be caught by the main loop \\
\begin{tikzpicture}
\umlclass[]{InteractiveCommandLineInterface.CommandError}{

}{
+ getMessage() : String
}
\end{tikzpicture}



\textbf{Superclasses and Interfaces}
\begin{itemize}
\item \lstinline|java.lang.Throwable|
\end{itemize}



\textbf{Constructors}
\begin{itemize}
\item \lstinline|public| \lstinline|InteractiveCommandLineInterface.CommandError|\lstinline|(String msg)|\\
Initializes the error
\begin{itemize}
\item \lstinline|msg|: The error message
\end{itemize}



\end{itemize}


\textbf{Methods}
\begin{itemize}
\item \lstinline|public String| \lstinline|getMessage|\lstinline|()|\\
Returns: The associated error message



\end{itemize}


