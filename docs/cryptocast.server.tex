\subsection{Package \lstinline!cryptocast.server!}
This package contains the server application used to send data to a specific group of people.

 The server implements the MVC pattern:
 \begin{itemize}
 \item \lstinline|Shell| is the view
 \item \lstinline|Controller| is the controller
 \item \lstinline|ServerData| is the model
 \end{itemize}

 It puts together the communication, cryptography and utility packages to implement its functionality.


\subsubsection{Class \lstinline|Controller|}
Deals with user-interactions and therefore changes data in the model if
 necessary. \\
\noindent\begin{minipage}[t]{5cm}
\vspace{0.3em}
\hspace*{2em}
\begin{tikzpicture}
\umlclass[]{Controller}{

}{
\umlstatic{+ start(databaseFile : File, listenAddr : SocketAddress, keyBroadcastIntervalSecs : int, serverFactory : NPServerFactory) : Controller} \\
\umlstatic{-- createNewData(t : int, serverFactory : NPServerFactory) : NaorPinkasServerData} \\
+ saveUserKeys(dir : File, users : Set<User<NPIdentity>, NPIdentity>) \\
+ saveDatabase() \\
+ getModel() : NaorPinkasServerData \\
+ reinitializeCrypto(t : int) \\
+ stream(in : InputStream) \\
+ stream(in : InputStream, maxBytesPerSec : long) \\
\umlstatic{-- startBroadcastEncryptionServer(data : NaorPinkasServerData, rawOut : MessageOutChannel, keyBroadcastIntervalSecs : int) : BroadcastEncryptionServer<NPIdentity>} \\
+ getDatabaseFile() : File \\
+ getT() : int \\
+ getListenAddress() : SocketAddress \\
+ streamSampleText() \\
+ streamAudio(file : File) \\
+ update(o : Observable, arg : Object)
}
\end{tikzpicture}
\vspace{0.3em}
\end{minipage}



\textbf{\sffamily Superclasses and Interfaces}
\begin{itemize}
\item \lstinline|java.util.Observer|
\end{itemize}


\textbf{\sffamily Constructors}
\begin{itemize}
\item \lstinline|private| \lstinline|Controller|\lstinline|(NaorPinkasServerData data, File databaseFile, MessageOutChannel rawOut, BroadcastEncryptionServer<NPIdentity> encServer, SocketAddress listenAddr, int keyBroadcastIntervalSecs, SwitchableInputStream switchableInput, SwitchableOutputStream switchableOutput, NPServerFactory serverFactory)| \\[-0.6em]




\end{itemize}


\textbf{\sffamily Methods}
\begin{itemize}
\item \lstinline|public static Controller| \lstinline|start|\lstinline|(File databaseFile, SocketAddress listenAddr, int keyBroadcastIntervalSecs, NPServerFactory serverFactory)|\\ \\[-0.6em]
Creates a Controller with the given parameters
\begin{itemize}
\item \lstinline|databaseFile|: The database file.
\item \lstinline|listenAddr|: The socket address to bind to.
\item \lstinline|keyBroadcastIntervalSecs|: the broadcast interval in seconds
\end{itemize}

\emph{Returns:} The controller

\item \lstinline|private static NaorPinkasServerData| \lstinline|createNewData|\lstinline|(int t, NPServerFactory serverFactory)| \\[-0.6em]




\item \lstinline|public void| \lstinline|saveUserKeys|\lstinline|(File dir, Set<User<NPIdentity>, NPIdentity> users)|\\ \\[-0.6em]
Saves the users's personal keys in a keyfile at the given directory.
\begin{itemize}
\item \lstinline|dir|: The directory to save the keyfiles in.
\item \lstinline|users|: The users who their personal keys will be saved.
\end{itemize}



\item \lstinline|public void| \lstinline|saveDatabase|\lstinline|()|\\ \\[-0.6em]
Saves the database.



\item \lstinline|public NaorPinkasServerData| \lstinline|getModel|\lstinline|()|\\ \\[-0.6em]
\emph{Returns:} the data.



\item \lstinline|public void| \lstinline|reinitializeCrypto|\lstinline|(int t)|\\ \\[-0.6em]
Reinitializes the cryptography. 
 All data of the current session will be lost forever!
\begin{itemize}
\item \lstinline|t|: the size of the polynomial.
\end{itemize}



\item \lstinline|public void| \lstinline|stream|\lstinline|(InputStream in)|\\ \\[-0.6em]
Streams the data.
\begin{itemize}
\item \lstinline|in|: The input stream.
\end{itemize}



\item \lstinline|public void| \lstinline|stream|\lstinline|(InputStream in, long maxBytesPerSec)|\\ \\[-0.6em]
Streams the data.
\begin{itemize}
\item \lstinline|in|: The input stream.
\item \lstinline|maxBytesPerSec|: Maximum bytes per second.
\end{itemize}



\item \lstinline|private static BroadcastEncryptionServer<NPIdentity>| \lstinline|startBroadcastEncryptionServer|\lstinline|(NaorPinkasServerData data, MessageOutChannel rawOut, int keyBroadcastIntervalSecs)| \\[-0.6em]




\item \lstinline|public File| \lstinline|getDatabaseFile|\lstinline|()|\\ \\[-0.6em]
\emph{Returns:} The database file.



\item \lstinline|public int| \lstinline|getT|\lstinline|()|\\ \\[-0.6em]
\emph{Returns:} The size of the polynomial.



\item \lstinline|public SocketAddress| \lstinline|getListenAddress|\lstinline|()|\\ \\[-0.6em]
\emph{Returns:} The socket listening address to bind to.



\item \lstinline|public void| \lstinline|streamSampleText|\lstinline|()|\\ \\[-0.6em]
Streams simple text.



\item \lstinline|public void| \lstinline|streamAudio|\lstinline|(File file)|\\ \\[-0.6em]
Streams audio.
\begin{itemize}
\item \lstinline|file|: The audio file from which the data is read.
\end{itemize}



\item \lstinline|public void| \lstinline|update|\lstinline|(Observable o, Object arg)| \\[-0.6em]




\end{itemize}

\subsubsection{Class \lstinline|User<ID>|}
Represents a user in our application. \\
\noindent\begin{minipage}[t]{5cm}
\vspace{0.3em}
\hspace*{2em}
\begin{tikzpicture}
\umlclass[]{User<ID>}{

}{
+ getName() : String \\
+ getIdentity() : ID
}
\end{tikzpicture}
\vspace{0.3em}
\end{minipage}

\begin{itemize}
\item \lstinline|<ID>|: The type of the user identities
\end{itemize}


\textbf{\sffamily Superclasses and Interfaces}
\begin{itemize}
\item \lstinline|java.io.Serializable|
\end{itemize}


\textbf{\sffamily Constructors}
\begin{itemize}
\item \lstinline|public| \lstinline|User|\lstinline|(String name, ID id)|\\ \\[-0.6em]
Creates a User with the given attributes.
\begin{itemize}
\item \lstinline|name|: The name of this user.
\item \lstinline|id|: The ID of this user.
\end{itemize}



\end{itemize}


\textbf{\sffamily Methods}
\begin{itemize}
\item \lstinline|public String| \lstinline|getName|\lstinline|()|\\ \\[-0.6em]
\emph{Returns:} The name of this user



\item \lstinline|public ID| \lstinline|getIdentity|\lstinline|()|\\ \\[-0.6em]
\emph{Returns:} The id of this user



\end{itemize}

\subsubsection{Class \lstinline|NaorPinkasServerData|}
Server data according to Naor-Pinkas. \\
\noindent\begin{minipage}[t]{5cm}
\vspace{0.3em}
\hspace*{2em}
\begin{tikzpicture}
\umlclass[]{NaorPinkasServerData}{

}{

}
\end{tikzpicture}
\vspace{0.3em}
\end{minipage}



\textbf{\sffamily Superclasses and Interfaces}
\begin{itemize}
\item \lstinline|cryptocast.server.ServerData<NPIdentity>|
\end{itemize}


\textbf{\sffamily Constructors}
\begin{itemize}
\item \lstinline|public| \lstinline|NaorPinkasServerData|\lstinline|(NPServerInterface npServer)|\\ \\[-0.6em]
Creates NaorPinkas server data with the given Naor-Pinkas server.
\begin{itemize}
\item \lstinline|npServer|: A Naor-Pinkas server.
\end{itemize}



\end{itemize}


\subsubsection{Class \lstinline|Shell|}
Implements the user interface of the server as an interactive console application. \\
\noindent\begin{minipage}[t]{5cm}
\vspace{0.3em}
\hspace*{2em}
\begin{tikzpicture}
\umlclass[]{Shell}{

}{
\# start() \\
\# performCommand(cmdName : String, args : String[]) \\
-- capitalizeCmdName(name : String) : String \\
\# cmdHelp(cmd : ShellCommand, args : String[]) \\
\# cmdInit(cmd : ShellCommand, args : String[]) \\
\# cmdUsers(cmd : ShellCommand, args : String[]) \\
\# cmdAdd(cmd : ShellCommand, args : String[]) \\
\# cmdRevoke(cmd : ShellCommand, args : String[]) \\
\# cmdUnrevoke(cmd : ShellCommand, args : String[]) \\
\# cmdSaveKeys(cmd : ShellCommand, args : String[]) \\
\# cmdStreamSampleText(cmd : ShellCommand, args : String[]) \\
\# cmdStreamStdin(cmd : ShellCommand, args : String[]) \\
\# cmdStreamMp3(cmd : ShellCommand, args : String[]) \\
-- getUser(name : String) : User<NPIdentity> \\
-- saveDatabase() \\
-- getModel() : ServerData<NPIdentity> \\
-- commandSyntaxError(cmd : ShellCommand) \\
\umlstatic{+ parseInt(str : String) : Optional<Integer>}
}
\end{tikzpicture}
\vspace{0.3em}
\end{minipage}



\textbf{\sffamily Superclasses and Interfaces}
\begin{itemize}
\item \lstinline|cryptocast.util.InteractiveCommandLineInterface|
\end{itemize}


\textbf{\sffamily Constructors}
\begin{itemize}
\item \lstinline|public| \lstinline|Shell|\lstinline|(BufferedReader in, PrintStream out, PrintStream err, Controller control)|\\ \\[-0.6em]
Creates a new Shell object with the given parameters.
\begin{itemize}
\item \lstinline|in|: The input stream
\item \lstinline|out|: Stream to write normal output to.
\item \lstinline|err|: Stream to write error messages to.
\end{itemize}



\end{itemize}


\textbf{\sffamily Methods}
\begin{itemize}
\item \lstinline|protected void| \lstinline|start|\lstinline|()| \\[-0.6em]




\item \lstinline|protected void| \lstinline|performCommand|\lstinline|(String cmdName, String[] args)| \\[-0.6em]




\item \lstinline|private String| \lstinline|capitalizeCmdName|\lstinline|(String name)| \\[-0.6em]




\item \lstinline|protected void| \lstinline|cmdHelp|\lstinline|(ShellCommand cmd, String[] args)|\\ \\[-0.6em]
Prints all commands this shell can perform with information about how to use them.



\item \lstinline|protected void| \lstinline|cmdInit|\lstinline|(ShellCommand cmd, String[] args)| \\[-0.6em]




\item \lstinline|protected void| \lstinline|cmdUsers|\lstinline|(ShellCommand cmd, String[] args)| \\[-0.6em]




\item \lstinline|protected void| \lstinline|cmdAdd|\lstinline|(ShellCommand cmd, String[] args)| \\[-0.6em]




\item \lstinline|protected void| \lstinline|cmdRevoke|\lstinline|(ShellCommand cmd, String[] args)| \\[-0.6em]




\item \lstinline|protected void| \lstinline|cmdUnrevoke|\lstinline|(ShellCommand cmd, String[] args)| \\[-0.6em]




\item \lstinline|protected void| \lstinline|cmdSaveKeys|\lstinline|(ShellCommand cmd, String[] args)| \\[-0.6em]




\item \lstinline|protected void| \lstinline|cmdStreamSampleText|\lstinline|(ShellCommand cmd, String[] args)| \\[-0.6em]




\item \lstinline|protected void| \lstinline|cmdStreamStdin|\lstinline|(ShellCommand cmd, String[] args)| \\[-0.6em]




\item \lstinline|protected void| \lstinline|cmdStreamMp3|\lstinline|(ShellCommand cmd, String[] args)| \\[-0.6em]




\item \lstinline|private User<NPIdentity>| \lstinline|getUser|\lstinline|(String name)| \\[-0.6em]




\item \lstinline|private void| \lstinline|saveDatabase|\lstinline|()| \\[-0.6em]




\item \lstinline|private ServerData<NPIdentity>| \lstinline|getModel|\lstinline|()| \\[-0.6em]




\item \lstinline|private void| \lstinline|commandSyntaxError|\lstinline|(ShellCommand cmd)| \\[-0.6em]




\item \lstinline|public static Optional<Integer>| \lstinline|parseInt|\lstinline|(String str)|\\ \\[-0.6em]
Parses the string.
\begin{itemize}
\item \lstinline|str|: The string to be parsed.
\end{itemize}

\emph{Returns:} An integer.

\end{itemize}

\subsubsection{Class \lstinline|ServerData<ID>|}
Contains the data which is managed by the controller and presented by the view. \\
\noindent\begin{minipage}[t]{5cm}
\vspace{0.3em}
\hspace*{2em}
\begin{tikzpicture}
\umlclass[]{ServerData<ID>}{

}{
+ createNewUser(name : String) : Optional<User<ID>, ID> \\
+ getUserByName(name : String) : Optional<User<ID>, ID> \\
+ getPersonalKey(user : User<ID>) : Optional<?, PrivateKey> \\
+ revoke(users : Set<User<ID>, ID>) \\
+ unrevoke(user : User<ID>) \\
+ isRevoked(user : User<ID>) : boolean \\
+ getUsers() : Set<User<ID>, ID>
}
\end{tikzpicture}
\vspace{0.3em}
\end{minipage}

\begin{itemize}
\item \lstinline|<ID>|: The type of the user identities
\end{itemize}


\textbf{\sffamily Superclasses and Interfaces}
\begin{itemize}
\item \lstinline|java.io.Serializable|
\end{itemize}


\textbf{\sffamily Constructors}
\begin{itemize}
\item \lstinline|public| \lstinline|ServerData|\lstinline|(BroadcastSchemeUserManager<ID> userManager, BroadcastSchemeKeyManager<ID> keyManager)|\\ \\[-0.6em]
Creates ServerData with the given parameter.
\begin{itemize}
\item \lstinline|userManager|: Manages the users.
\item \lstinline|keyManager|: Manages the keys.
\end{itemize}



\end{itemize}


\textbf{\sffamily Methods}
\begin{itemize}
\item \lstinline|public Optional<User<ID>, ID>| \lstinline|createNewUser|\lstinline|(String name)|\\ \\[-0.6em]
Creates and saves a new user by name.
\begin{itemize}
\item \lstinline|name|: The user's name.
\end{itemize}

\emph{Returns:} The new user if it has been added successfully or absent
         if a user with the same name already existed.

\item \lstinline|public Optional<User<ID>, ID>| \lstinline|getUserByName|\lstinline|(String name)|\\ \\[-0.6em]
Retrieves a user by name.
\begin{itemize}
\item \lstinline|name|: The user's name
\end{itemize}

\emph{Returns:} A user instance, if it was found, or absent otherwise

\item \lstinline|public Optional<?, PrivateKey>| \lstinline|getPersonalKey|\lstinline|(User<ID> user)|\\ \\[-0.6em]
Retrieves a user's personal key.
\begin{itemize}
\item \lstinline|user|: The user object.
\end{itemize}

\emph{Returns:} The private key.

\item \lstinline|public void| \lstinline|revoke|\lstinline|(Set<User<ID>, ID> users)|\\ \\[-0.6em]
Revokes users.
\begin{itemize}
\item \lstinline|users|: The users to revoke.
\end{itemize}

\emph{Returns:} <code>true</code>, if the set of revoked users changed or <code>false</code> otherwise.

\item \lstinline|public void| \lstinline|unrevoke|\lstinline|(User<ID> user)|\\ \\[-0.6em]
Authorizes a user.
\begin{itemize}
\item \lstinline|user|: The user to authorize.
\end{itemize}

\emph{Returns:} <code>true</code>, if the set of revoked users changed or <code>false</code> otherwise.

\item \lstinline|public boolean| \lstinline|isRevoked|\lstinline|(User<ID> user)|\\ \\[-0.6em]
Checks whether the user is revoked.
\begin{itemize}
\item \lstinline|user|: The user.
\end{itemize}

\emph{Returns:} <code>true</code>, if the user was revoked or <code>false</code> otherwise.

\item \lstinline|public Set<User<ID>, ID>| \lstinline|getUsers|\lstinline|()|\\ \\[-0.6em]
\emph{Returns:} all users.



\end{itemize}

\subsubsection{Class \lstinline|ShellCommand|}
Represents the commands available in the shell. \\
\noindent\begin{minipage}[t]{5cm}
\vspace{0.3em}
\hspace*{2em}
\begin{tikzpicture}
\umlclass[]{ShellCommand}{

}{
+ getName() : String \\
+ getSyntax() : String \\
+ getShortDesc() : String \\
+ getLongDesc() : String
}
\end{tikzpicture}
\vspace{0.3em}
\end{minipage}




\textbf{\sffamily Constructors}
\begin{itemize}
\item \lstinline|public| \lstinline|ShellCommand|\lstinline|(String name, String syntax, String shortDesc, String longDesc)|\\ \\[-0.6em]
Creates a new Command with the given values
\begin{itemize}
\item \lstinline|name|: The name of the command
\item \lstinline|syntax|: The syntax of the command
\end{itemize}



\item \lstinline|public| \lstinline|ShellCommand|\lstinline|(String name, String syntax, String shortDesc)|\\ \\[-0.6em]
Creates a new Command with the given values.
\begin{itemize}
\item \lstinline|name|: The name of the command.
\item \lstinline|syntax|: The syntax of the command.
\end{itemize}



\end{itemize}


\textbf{\sffamily Methods}
\begin{itemize}
\item \lstinline|public String| \lstinline|getName|\lstinline|()|\\ \\[-0.6em]
\emph{Returns:} The command name



\item \lstinline|public String| \lstinline|getSyntax|\lstinline|()|\\ \\[-0.6em]
\emph{Returns:} The command syntax



\item \lstinline|public String| \lstinline|getShortDesc|\lstinline|()|\\ \\[-0.6em]
\emph{Returns:} The short command description



\item \lstinline|public String| \lstinline|getLongDesc|\lstinline|()|\\ \\[-0.6em]
\emph{Returns:} The long command description



\end{itemize}

\subsubsection{Class \lstinline|LogbackUtils|}
Helper functions to work with Logback. \\
\noindent\begin{minipage}[t]{5cm}
\vspace{0.3em}
\hspace*{2em}
\begin{tikzpicture}
\umlclass[]{LogbackUtils}{

}{
\umlstatic{+ removeAllAppenders()} \\
\umlstatic{+ setRootLogLevel(level : Level)} \\
\umlstatic{+ addFileAppender(logFile : File, level : Level, pattern : String)} \\
\umlstatic{+ addFileAppender(logFile : File, level : Level)} \\
\umlstatic{+ addStderrLogger(level : Level, pattern : String)} \\
\umlstatic{+ addStderrLogger(level : Level)} \\
\umlstatic{+ makeThresholdFilter(level : Level) : ThresholdFilter}
}
\end{tikzpicture}
\vspace{0.3em}
\end{minipage}





\textbf{\sffamily Methods}
\begin{itemize}
\item \lstinline|public static void| \lstinline|removeAllAppenders|\lstinline|()| \\[-0.6em]




\item \lstinline|public static void| \lstinline|setRootLogLevel|\lstinline|(Level level)| \\[-0.6em]




\item \lstinline|public static void| \lstinline|addFileAppender|\lstinline|(File logFile, Level level, String pattern)| \\[-0.6em]




\item \lstinline|public static void| \lstinline|addFileAppender|\lstinline|(File logFile, Level level)| \\[-0.6em]




\item \lstinline|public static void| \lstinline|addStderrLogger|\lstinline|(Level level, String pattern)| \\[-0.6em]




\item \lstinline|public static void| \lstinline|addStderrLogger|\lstinline|(Level level)| \\[-0.6em]




\item \lstinline|public static ThresholdFilter| \lstinline|makeThresholdFilter|\lstinline|(Level level)| \\[-0.6em]




\end{itemize}

\subsubsection{Class \lstinline|OptParse|}
A Command line option parser. \\
\noindent\begin{minipage}[t]{5cm}
\vspace{0.3em}
\hspace*{2em}
\begin{tikzpicture}
\umlclass[]{OptParse}{

}{
\umlstatic{+ parseArgs(jcOpts : T, programName : String, argv : String[]) : T}
}
\end{tikzpicture}
\vspace{0.3em}
\end{minipage}





\textbf{\sffamily Methods}
\begin{itemize}
\item \lstinline|public static T| \lstinline|parseArgs|\lstinline|(T jcOpts, String programName, String[] argv)|\\ \\[-0.6em]
Parses the program's arguments
\begin{itemize}
\item \lstinline|jcOpts|: jcommander options for parsing.
\item \lstinline|programName|: Program name.
\item \lstinline|argv|: Program arguments.
\end{itemize}

\emph{Returns:} parsed options.

\end{itemize}

\subsubsection{Class \lstinline|OptParse.WithHelp|}
A base class for option beans the provide a help flag \\
\noindent\begin{minipage}[t]{5cm}
\vspace{0.3em}
\hspace*{2em}
\begin{tikzpicture}
\umlclass[]{OptParse.WithHelp}{

}{

}
\end{tikzpicture}
\vspace{0.3em}
\end{minipage}






