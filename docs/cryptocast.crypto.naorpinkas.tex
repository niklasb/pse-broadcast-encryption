\subsection{Package \lstinline!cryptocast.crypto.naorpinkas!}
An implementation of the Naor-Pinkas protocol based on the primitives provided by the crypto
 subpackage.


\subsubsection{Class \lstinline|NPServer<T, G>|}
A server in the Naor-Pinkas broadcast encryption scheme. It knows
 the entire polynomial and therefore all the private keys of its users. \\
\noindent\begin{minipage}[t]{5cm}
\vspace{0.3em}
\hspace*{2em}
\begin{tikzpicture}
\umlclass[type=abstract]{NPServer<T, G>}{

}{
+ getT() : int \\
+ getContext() : NPServerContext<T, G> \\
+ encrypt(secret : byte[]) : byte[] \\
+ getIdentity(i : int) : NPIdentity \\
+ revoke(ids : Set<NPIdentity>) : boolean \\
+ revoke(id : NPIdentity) : boolean \\
+ unrevoke(id : NPIdentity) : boolean \\
+ isRevoked(id : NPIdentity) : boolean \\
+ getPersonalKey(id : NPIdentity) : Optional<NPKey<T, G>, T, G>
}
\end{tikzpicture}
\vspace{0.3em}
\end{minipage}

\begin{itemize}
\item \lstinline|<T>|: 
\item \lstinline|<G>|: 
\end{itemize}


\textbf{\sffamily Superclasses and Interfaces}
\begin{itemize}
\item \lstinline|java.util.Observable|
\item \lstinline|cryptocast.crypto.naorpinkas.NPServerInterface|
\item \lstinline|java.io.Serializable|
\end{itemize}


\textbf{\sffamily Constructors}
\begin{itemize}
\item \lstinline|protected| \lstinline|NPServer|\lstinline|(NPServerContext<T, G> context)| \\[-0.6em]




\end{itemize}


\textbf{\sffamily Methods}
\begin{itemize}
\item \lstinline|public int| \lstinline|getT|\lstinline|()| \\[-0.6em]




\item \lstinline|public NPServerContext<T, G>| \lstinline|getContext|\lstinline|()|\\ \\[-0.6em]
\emph{Returns:} The NP server context.



\item \lstinline|public byte[]| \lstinline|encrypt|\lstinline|(byte[] secret)| \\[-0.6em]




\item \lstinline|public NPIdentity| \lstinline|getIdentity|\lstinline|(int i)|\\ \\[-0.6em]
\emph{Returns:} The identity with the given index.
\begin{itemize}
\item \lstinline|i|: An index.
\end{itemize}



\item \lstinline|public boolean| \lstinline|revoke|\lstinline|(Set<NPIdentity> ids)|\\ \\[-0.6em]
Revokes multiple users.

\emph{Returns:} true, if the set of revoked users changed or false otherwise

\item \lstinline|public boolean| \lstinline|revoke|\lstinline|(NPIdentity id)|\\ \\[-0.6em]
Revokes a user.
\begin{itemize}
\item \lstinline|id|: The identity of the user
\end{itemize}

\emph{Returns:} true, if the set of revoked users changed or false otherwise

\item \lstinline|public boolean| \lstinline|unrevoke|\lstinline|(NPIdentity id)|\\ \\[-0.6em]
Authorizes a user.
\begin{itemize}
\item \lstinline|id|: The identity of the user
\end{itemize}

\emph{Returns:} true, if the set of revoked users changed or false otherwise

\item \lstinline|public boolean| \lstinline|isRevoked|\lstinline|(NPIdentity id)|\\ \\[-0.6em]
\emph{Returns:} Whether the user is revoked.
\begin{itemize}
\item \lstinline|id|: The identity of the user
\end{itemize}



\item \lstinline|public Optional<NPKey<T, G>, T, G>| \lstinline|getPersonalKey|\lstinline|(NPIdentity id)|\\ \\[-0.6em]
\emph{Returns:} The private key of the user or absent if no
 such user exists
\begin{itemize}
\item \lstinline|id|: The identity to look up
\end{itemize}



\end{itemize}

\subsubsection{Class \lstinline|ECNPServerFactory|}
Creates instances of NP servers that use the variant based on elliptic
 curves. \\
\noindent\begin{minipage}[t]{5cm}
\vspace{0.3em}
\hspace*{2em}
\begin{tikzpicture}
\umlclass[]{ECNPServerFactory}{

}{
+ construct(t : int, g : EllipticCurveGroup<BigInteger, Point, EllipticCurveOverFp>) : ECNPServer \\
+ construct(t : int) : ECNPServer
}
\end{tikzpicture}
\vspace{0.3em}
\end{minipage}



\textbf{\sffamily Superclasses and Interfaces}
\begin{itemize}
\item \lstinline|cryptocast.crypto.naorpinkas.NPServerFactory|
\end{itemize}



\textbf{\sffamily Methods}
\begin{itemize}
\item \lstinline|public ECNPServer| \lstinline|construct|\lstinline|(int t, EllipticCurveGroup<BigInteger, Point, EllipticCurveOverFp> g)|\\ \\[-0.6em]
\emph{Returns:} An NP server instance.
\begin{itemize}
\item \lstinline|t|: The parameter $t$ of the NP scheme.
\item \lstinline|g|: An EC group.
\end{itemize}



\item \lstinline|public ECNPServer| \lstinline|construct|\lstinline|(int t)| \\[-0.6em]




\end{itemize}

\subsubsection{Class \lstinline|SchnorrNPClient|}
A client of the NP variant that uses a Schnorr group. \\
\noindent\begin{minipage}[t]{5cm}
\vspace{0.3em}
\hspace*{2em}
\begin{tikzpicture}
\umlclass[]{SchnorrNPClient}{

}{

}
\end{tikzpicture}
\vspace{0.3em}
\end{minipage}



\textbf{\sffamily Superclasses and Interfaces}
\begin{itemize}
\item \lstinline|cryptocast.crypto.naorpinkas.NPClient<BigInteger, SchnorrGroup>|
\end{itemize}


\textbf{\sffamily Constructors}
\begin{itemize}
\item \lstinline|public| \lstinline|SchnorrNPClient|\lstinline|(NPKey<BigInteger, SchnorrGroup> key)|\\ \\[-0.6em]
Initializes an instance.
\begin{itemize}
\item \lstinline|key|: The personal key of the user
\end{itemize}



\end{itemize}


\subsubsection{Class \lstinline|ECNPClient|}
A client of the NP variant that uses elliptic curves. \\
\noindent\begin{minipage}[t]{5cm}
\vspace{0.3em}
\hspace*{2em}
\begin{tikzpicture}
\umlclass[]{ECNPClient}{

}{

}
\end{tikzpicture}
\vspace{0.3em}
\end{minipage}



\textbf{\sffamily Superclasses and Interfaces}
\begin{itemize}
\item \lstinline|cryptocast.crypto.naorpinkas.NPClient<Point, EllipticCurveGroup<BigInteger, Point, EllipticCurveOverFp>, BigInteger, Point, EllipticCurveOverFp>|
\end{itemize}


\textbf{\sffamily Constructors}
\begin{itemize}
\item \lstinline|public| \lstinline|ECNPClient|\lstinline|(NPKey<Point, EllipticCurveGroup<BigInteger, Point, EllipticCurveOverFp>, BigInteger, Point, EllipticCurveOverFp> key)|\\ \\[-0.6em]
Initializes an instance.
\begin{itemize}
\item \lstinline|key|: The personal key of the user
\end{itemize}



\end{itemize}


\subsubsection{Class \lstinline|NPClient<T, G>|}
A client in the Naor-Pinkas broadcast encryption scheme. \\
\noindent\begin{minipage}[t]{5cm}
\vspace{0.3em}
\hspace*{2em}
\begin{tikzpicture}
\umlclass[type=abstract]{NPClient<T, G>}{

}{
+ getGroup() : G \\
+ unpackMessage(msg : byte[]) : NPMessage<T, G> \\
+ decrypt(cipher : byte[]) : byte[]
}
\end{tikzpicture}
\vspace{0.3em}
\end{minipage}

\begin{itemize}
\item \lstinline|<T>|: 
\item \lstinline|<G>|: 
\end{itemize}


\textbf{\sffamily Superclasses and Interfaces}
\begin{itemize}
\item \lstinline|cryptocast.crypto.Decryptor<byte[]>|
\end{itemize}


\textbf{\sffamily Constructors}
\begin{itemize}
\item \lstinline|protected| \lstinline|NPClient|\lstinline|(NPKey<T, G> key)|\\ \\[-0.6em]
Initializes a Naor-Pinkas broadcast client.
\begin{itemize}
\item \lstinline|key|: The personal key used to reconstruct a secret from the stream.
\end{itemize}



\end{itemize}


\textbf{\sffamily Methods}
\begin{itemize}
\item \lstinline|public G| \lstinline|getGroup|\lstinline|()|\\ \\[-0.6em]
\emph{Returns:} The underlying group.



\item \lstinline|public NPMessage<T, G>| \lstinline|unpackMessage|\lstinline|(byte[] msg)|\\ \\[-0.6em]
Parse a NP message.



\item \lstinline|public byte[]| \lstinline|decrypt|\lstinline|(byte[] cipher)| \\[-0.6em]




\end{itemize}

\subsubsection{Class \lstinline|NPServerContext<T, G>|}
Represents the state of a NP server \\
\noindent\begin{minipage}[t]{5cm}
\vspace{0.3em}
\hspace*{2em}
\begin{tikzpicture}
\umlclass[]{NPServerContext<T, G>}{

}{
+ getLagrange() : LagrangeInterpolation<BigInteger> \\
+ getGroup() : G \\
+ getKeyGen() : Generator<NPKey<T, G>, T, G> \\
+ getPoly() : Polynomial<BigInteger> \\
+ getT() : int \\
\umlstatic{+ generate(t : int, group : G) : NPServerContext<T, G>}
}
\end{tikzpicture}
\vspace{0.3em}
\end{minipage}

\begin{itemize}
\item \lstinline|<T>|: 
\item \lstinline|<G>|: 
\end{itemize}


\textbf{\sffamily Superclasses and Interfaces}
\begin{itemize}
\item \lstinline|java.io.Serializable|
\end{itemize}


\textbf{\sffamily Constructors}
\begin{itemize}
\item \lstinline|public| \lstinline|NPServerContext|\lstinline|(int t, G group, Generator<NPKey<T, G>, T, G> keyGen, Polynomial<BigInteger> poly, LagrangeInterpolation<BigInteger> lagrange)| \\[-0.6em]




\end{itemize}


\textbf{\sffamily Methods}
\begin{itemize}
\item \lstinline|public LagrangeInterpolation<BigInteger>| \lstinline|getLagrange|\lstinline|()| \\[-0.6em]




\item \lstinline|public G| \lstinline|getGroup|\lstinline|()| \\[-0.6em]




\item \lstinline|public Generator<NPKey<T, G>, T, G>| \lstinline|getKeyGen|\lstinline|()| \\[-0.6em]




\item \lstinline|public Polynomial<BigInteger>| \lstinline|getPoly|\lstinline|()| \\[-0.6em]




\item \lstinline|public int| \lstinline|getT|\lstinline|()|\\ \\[-0.6em]
\emph{Returns:} The degree of the polynomial.



\item \lstinline|public static NPServerContext<T, G>| \lstinline|generate|\lstinline|(int t, G group)|\\ \\[-0.6em]
Generates a NP server context.
\begin{itemize}
\item \lstinline|t|: The degree of the polynomial.
\item \lstinline|group|: The NP group.
\end{itemize}

\emph{Returns:} Naor-pinkas server instance.

\end{itemize}

\subsubsection{Class \lstinline|NPIdentity|}
An identity in the Naor-Pinkas broadcast encryption scheme \\
\noindent\begin{minipage}[t]{5cm}
\vspace{0.3em}
\hspace*{2em}
\begin{tikzpicture}
\umlclass[]{NPIdentity}{

}{
+ hashCode() : int \\
+ equals(other : Object) : boolean
}
\end{tikzpicture}
\vspace{0.3em}
\end{minipage}



\textbf{\sffamily Superclasses and Interfaces}
\begin{itemize}
\item \lstinline|java.io.Serializable|
\end{itemize}


\textbf{\sffamily Constructors}
\begin{itemize}
\item \lstinline|protected| \lstinline|NPIdentity|\lstinline|(BigInteger i)|\\ \\[-0.6em]
initializes an instance.
\begin{itemize}
\item \lstinline|i|: The identity code.
\end{itemize}



\end{itemize}


\textbf{\sffamily Methods}
\begin{itemize}
\item \lstinline|public int| \lstinline|hashCode|\lstinline|()| \\[-0.6em]




\item \lstinline|public boolean| \lstinline|equals|\lstinline|(Object other)| \\[-0.6em]




\end{itemize}

\subsubsection{Class \lstinline|NPMessage<T, G>|}
This class represents a NP message sent from the
 server to the clients. \\
\noindent\begin{minipage}[t]{5cm}
\vspace{0.3em}
\hspace*{2em}
\begin{tikzpicture}
\umlclass[]{NPMessage<T, G>}{

}{
+ getT() : int \\
+ getGroup() : G \\
+ getEncryptedSecret() : byte[] \\
+ getR() : BigInteger \\
+ getShares() : ImmutableList<NPShare<T, G>, T, G> \\
+ getLagrangeCoeffs() : ImmutableList<BigInteger>
}
\end{tikzpicture}
\vspace{0.3em}
\end{minipage}

\begin{itemize}
\item \lstinline|<T>|: 
\item \lstinline|<G>|: 
\end{itemize}



\textbf{\sffamily Constructors}
\begin{itemize}
\item \lstinline|public| \lstinline|NPMessage|\lstinline|(int t, BigInteger r, byte[] encryptedSecret, G group, List<BigInteger> lagrangeCoefficients, List<NPShare<T, G>, T, G> shares)|\\ \\[-0.6em]
Creates a new instance of NaorPinkasMessage with the given parameters.
\begin{itemize}
\item \lstinline|t|: The degree of the polynomial.
\item \lstinline|r|: The $r$ value.
\item \lstinline|encryptedSecret|: A value, from which we can restore the secret
                        together with the value $g^{r \cdot P(0)}$
\item \lstinline|shares|: The shares.
\end{itemize}



\end{itemize}


\textbf{\sffamily Methods}
\begin{itemize}
\item \lstinline|public int| \lstinline|getT|\lstinline|()|\\ \\[-0.6em]
\emph{Returns:} The value $t$.



\item \lstinline|public G| \lstinline|getGroup|\lstinline|()|\\ \\[-0.6em]
\emph{Returns:} The group.



\item \lstinline|public byte[]| \lstinline|getEncryptedSecret|\lstinline|()|\\ \\[-0.6em]
\emph{Returns:} The encrypted secret.



\item \lstinline|public BigInteger| \lstinline|getR|\lstinline|()|\\ \\[-0.6em]
\emph{Returns:} The value $r$.



\item \lstinline|public ImmutableList<NPShare<T, G>, T, G>| \lstinline|getShares|\lstinline|()|\\ \\[-0.6em]
\emph{Returns:} The shares.



\item \lstinline|public ImmutableList<BigInteger>| \lstinline|getLagrangeCoeffs|\lstinline|()|\\ \\[-0.6em]
\emph{Returns:} The lagrange coefficients belonging to the shares.



\end{itemize}

\subsubsection{Class \lstinline|NPShare<T, G>|}
A share in the Naor-Pinkas broadcast encryption scheme. It consists of a tuple
 $(r, I, g^{r P(I)})$. $t + 1$ distinct shares of this form are sufficient to restore the
 value $g^{r P(0)}$, where $t$ is the degree of the polynomial $P$. \\
\noindent\begin{minipage}[t]{5cm}
\vspace{0.3em}
\hspace*{2em}
\begin{tikzpicture}
\umlclass[]{NPShare<T, G>}{

}{
+ getGroup() : G \\
+ getI() : BigInteger \\
+ getGRPI() : T \\
+ getIdentity() : NPIdentity \\
+ compareTo(other : NPShare<T, G>) : int \\
+ equals(other\_ : Object) : boolean \\
\umlstatic{+ getXsFromShares(shares : List<NPShare<T, G>, T, G>) : ImmutableList<BigInteger>}
}
\end{tikzpicture}
\vspace{0.3em}
\end{minipage}

\begin{itemize}
\item \lstinline|<T>|: 
\item \lstinline|<G>|: 
\end{itemize}


\textbf{\sffamily Superclasses and Interfaces}
\begin{itemize}
\item \lstinline|java.lang.Comparable<NPShare<T, G>, T, G>|
\end{itemize}


\textbf{\sffamily Constructors}
\begin{itemize}
\item \lstinline|public| \lstinline|NPShare|\lstinline|(BigInteger i, T grpi, G group)|\\ \\[-0.6em]
Creates a new NP share from a tuple $(I, g^{r \cdot P(I)})$
\begin{itemize}
\item \lstinline|i|: An integer $i < q$.
\item \lstinline|grpi|: The value $g^{r \cdot P(I)}) \in G$.
\item \lstinline|group|: The NP group.
\end{itemize}



\end{itemize}


\textbf{\sffamily Methods}
\begin{itemize}
\item \lstinline|public G| \lstinline|getGroup|\lstinline|()|\\ \\[-0.6em]
\emph{Returns:} The cyclic group used by NP.



\item \lstinline|public BigInteger| \lstinline|getI|\lstinline|()|\\ \\[-0.6em]
\emph{Returns:} The integer $I$.



\item \lstinline|public T| \lstinline|getGRPI|\lstinline|()|\\ \\[-0.6em]
\emph{Returns:} $g^{r \cdot P(I)}$



\item \lstinline|public NPIdentity| \lstinline|getIdentity|\lstinline|()|\\ \\[-0.6em]
\emph{Returns:} The identity



\item \lstinline|public int| \lstinline|compareTo|\lstinline|(NPShare<T, G> other)| \\[-0.6em]




\item \lstinline|public boolean| \lstinline|equals|\lstinline|(Object other_)| \\[-0.6em]




\item \lstinline|public static ImmutableList<BigInteger>| \lstinline|getXsFromShares|\lstinline|(List<NPShare<T, G>, T, G> shares)|\\ \\[-0.6em]
Returns a list of points from shares.
\begin{itemize}
\item \lstinline|shares|: The list of shares.
\end{itemize}

\emph{Returns:} The points $I_i$ extracted from the given shares

\end{itemize}

\subsubsection{Class \lstinline|NPKeyGenerator<T, G>|}
A class generating NP keys. \\
\noindent\begin{minipage}[t]{5cm}
\vspace{0.3em}
\hspace*{2em}
\begin{tikzpicture}
\umlclass[]{NPKeyGenerator<T, G>}{

}{
+ get(i : int) : NPKey<T, G> \\
+ getRange(a : int, b : int) : ImmutableList<NPKey<T, G>, T, G> \\
+ getKey(id : NPIdentity) : Optional<NPKey<T, G>, T, G>
}
\end{tikzpicture}
\vspace{0.3em}
\end{minipage}

\begin{itemize}
\item \lstinline|<T>|: 
\item \lstinline|<G>|: 
\end{itemize}


\textbf{\sffamily Superclasses and Interfaces}
\begin{itemize}
\item \lstinline|cryptocast.util.Generator<NPKey<T, G>, T, G>|
\item \lstinline|java.io.Serializable|
\end{itemize}


\textbf{\sffamily Constructors}
\begin{itemize}
\item \lstinline|public| \lstinline|NPKeyGenerator|\lstinline|(SecureRandom rnd, G group, Polynomial<BigInteger> poly)|\\ \\[-0.6em]
Creates a new instance of NaorPinkasKeyGenerator with the given parameters.
\begin{itemize}
\item \lstinline|rnd|: The secure random number generator.
\item \lstinline|group|: The NP group.
\item \lstinline|poly|: The polynomial.
\end{itemize}



\end{itemize}


\textbf{\sffamily Methods}
\begin{itemize}
\item \lstinline|public NPKey<T, G>| \lstinline|get|\lstinline|(int i)| \\[-0.6em]




\item \lstinline|public ImmutableList<NPKey<T, G>, T, G>| \lstinline|getRange|\lstinline|(int a, int b)| \\[-0.6em]




\item \lstinline|public Optional<NPKey<T, G>, T, G>| \lstinline|getKey|\lstinline|(NPIdentity id)|\\ \\[-0.6em]
\emph{Returns:} the key for the given identity.
\begin{itemize}
\item \lstinline|id|: An identity.
\end{itemize}



\end{itemize}

\subsubsection{Class \lstinline|NPShareCombinator<T, G>|}
Allows to restore a number from a sufficient number of Naor-Pinkas shares. \\
\noindent\begin{minipage}[t]{5cm}
\vspace{0.3em}
\hspace*{2em}
\begin{tikzpicture}
\umlclass[]{NPShareCombinator<T, G>}{

}{
+ restore(shares : List<NPShare<T, G>, T, G>, lagrange : LagrangeInterpolation<BigInteger>) : Optional<T> \\
+ hasMissingShares(shares : List<NPShare<T, G>, T, G>) : boolean \\
+ hasRedundantShares(shares : List<NPShare<T, G>, T, G>) : boolean
}
\end{tikzpicture}
\vspace{0.3em}
\end{minipage}

\begin{itemize}
\item \lstinline|<T>|: 
\item \lstinline|<G>|: 
\end{itemize}



\textbf{\sffamily Constructors}
\begin{itemize}
\item \lstinline|public| \lstinline|NPShareCombinator|\lstinline|(int t, G group)| \\[-0.6em]




\end{itemize}


\textbf{\sffamily Methods}
\begin{itemize}
\item \lstinline|public Optional<T>| \lstinline|restore|\lstinline|(List<NPShare<T, G>, T, G> shares, LagrangeInterpolation<BigInteger> lagrange)|\\ \\[-0.6em]
Restores a secret from several Naor-Pinkas shares.
\begin{itemize}
\item \lstinline|shares|: The shares
\item \lstinline|lagrange|: The precomputed Lagrange coefficients
\end{itemize}

\emph{Returns:} The reconstructed secret or absent if the information represented
 by the given shares is insufficient to restore it.

\item \lstinline|public boolean| \lstinline|hasMissingShares|\lstinline|(List<NPShare<T, G>, T, G> shares)|\\ \\[-0.6em]
Checks whether missing shares exist.
\begin{itemize}
\item \lstinline|shares|: The shares to check.
\end{itemize}

\emph{Returns:} \lstinline|true| if there are any missing shares, \lstinline|false| otherwise.

\item \lstinline|public boolean| \lstinline|hasRedundantShares|\lstinline|(List<NPShare<T, G>, T, G> shares)|\\ \\[-0.6em]
Checks whether redundant shares exist.
\begin{itemize}
\item \lstinline|shares|: The shares.
\end{itemize}

\emph{Returns:} \lstinline|true| if there are any redundant shares, \lstinline|false| otherwise.

\end{itemize}

\subsubsection{Class \lstinline|NPKey<T, G>|}
A user's personal key in the Naor-Pinkas broadcast encryption scheme.
 It consists of a value $I$ and the value $P(I)$, which is the value of
 the Naor-Pinkas polynomial at $I$. \\
\noindent\begin{minipage}[t]{5cm}
\vspace{0.3em}
\hspace*{2em}
\begin{tikzpicture}
\umlclass[]{NPKey<T, G>}{

}{
+ getGroup() : G \\
+ getShare(r : BigInteger, gr : T) : NPShare<T, G> \\
+ getShare(r : BigInteger) : NPShare<T, G> \\
+ getIdentity() : NPIdentity \\
+ getAlgorithm() : String \\
+ getEncoded() : byte[] \\
+ getFormat() : String
}
\end{tikzpicture}
\vspace{0.3em}
\end{minipage}

\begin{itemize}
\item \lstinline|<T>|: 
\item \lstinline|<G>|: 
\end{itemize}


\textbf{\sffamily Superclasses and Interfaces}
\begin{itemize}
\item \lstinline|java.security.PrivateKey|
\end{itemize}


\textbf{\sffamily Constructors}
\begin{itemize}
\item \lstinline|protected| \lstinline|NPKey|\lstinline|(BigInteger i, BigInteger pi, G group)|\\ \\[-0.6em]
Creates a new instance of NaorPinkasPersonalKey with the given parameters.
\begin{itemize}
\item \lstinline|i|: A point $I$.
\item \lstinline|pi|: $P(I)$, the value of the polynomial at $I$.
\end{itemize}



\end{itemize}


\textbf{\sffamily Methods}
\begin{itemize}
\item \lstinline|public G| \lstinline|getGroup|\lstinline|()|\\ \\[-0.6em]
\emph{Returns:} The NP group.



\item \lstinline|public NPShare<T, G>| \lstinline|getShare|\lstinline|(BigInteger r, T gr)|\\ \\[-0.6em]
\emph{Returns:} a NP share constructed from this key and the given values.
\begin{itemize}
\item \lstinline|r|: The integer $r$.
\item \lstinline|gr|: The value $g^r$.
\end{itemize}



\item \lstinline|public NPShare<T, G>| \lstinline|getShare|\lstinline|(BigInteger r)|\\ \\[-0.6em]
\emph{Returns:} a NP share constructed from this key and the given $r$.
\begin{itemize}
\item \lstinline|r|: The integer $r$
\end{itemize}



\item \lstinline|public NPIdentity| \lstinline|getIdentity|\lstinline|()|\\ \\[-0.6em]
\emph{Returns:} the NP identity.



\item \lstinline|public String| \lstinline|getAlgorithm|\lstinline|()|\\ \\[-0.6em]
\emph{Returns:} the name of the algorithm associated with this key.



\item \lstinline|public byte[]| \lstinline|getEncoded|\lstinline|()|\\ \\[-0.6em]
\emph{Returns:} null



\item \lstinline|public String| \lstinline|getFormat|\lstinline|()|\\ \\[-0.6em]
\emph{Returns:} null



\end{itemize}

\subsubsection{Class \lstinline|ECNPServer|}
The server of the NP variant the uses elliptic curves. \\
\noindent\begin{minipage}[t]{5cm}
\vspace{0.3em}
\hspace*{2em}
\begin{tikzpicture}
\umlclass[]{ECNPServer}{

}{

}
\end{tikzpicture}
\vspace{0.3em}
\end{minipage}



\textbf{\sffamily Superclasses and Interfaces}
\begin{itemize}
\item \lstinline|cryptocast.crypto.naorpinkas.NPServer<Point, EllipticCurveGroup<BigInteger, Point, EllipticCurveOverFp>, BigInteger, Point, EllipticCurveOverFp>|
\end{itemize}


\textbf{\sffamily Constructors}
\begin{itemize}
\item \lstinline|protected| \lstinline|ECNPServer|\lstinline|(NPServerContext<Point, EllipticCurveGroup<BigInteger, Point, EllipticCurveOverFp>, BigInteger, Point, EllipticCurveOverFp> ctx)| \\[-0.6em]




\end{itemize}


\subsubsection{Class \lstinline|SchnorrNPServerFactory|}
Creates instances of NP servers that use the variant based on a Schnorr
 group. \\
\noindent\begin{minipage}[t]{5cm}
\vspace{0.3em}
\hspace*{2em}
\begin{tikzpicture}
\umlclass[]{SchnorrNPServerFactory}{

}{
+ construct(t : int, g : SchnorrGroup) : SchnorrNPServer \\
+ construct(t : int) : SchnorrNPServer
}
\end{tikzpicture}
\vspace{0.3em}
\end{minipage}



\textbf{\sffamily Superclasses and Interfaces}
\begin{itemize}
\item \lstinline|cryptocast.crypto.naorpinkas.NPServerFactory|
\end{itemize}



\textbf{\sffamily Methods}
\begin{itemize}
\item \lstinline|public SchnorrNPServer| \lstinline|construct|\lstinline|(int t, SchnorrGroup g)|\\ \\[-0.6em]
\emph{Returns:} An NP server instance.
\begin{itemize}
\item \lstinline|t|: The parameter $t$ of the NP scheme.
\item \lstinline|g|: A group instance.
\end{itemize}



\item \lstinline|public SchnorrNPServer| \lstinline|construct|\lstinline|(int t)| \\[-0.6em]




\end{itemize}

\subsubsection{Class \lstinline|SchnorrNPServer|}
The server of the NP variant the uses a Schnorr group. \\
\noindent\begin{minipage}[t]{5cm}
\vspace{0.3em}
\hspace*{2em}
\begin{tikzpicture}
\umlclass[]{SchnorrNPServer}{

}{

}
\end{tikzpicture}
\vspace{0.3em}
\end{minipage}



\textbf{\sffamily Superclasses and Interfaces}
\begin{itemize}
\item \lstinline|cryptocast.crypto.naorpinkas.NPServer<BigInteger, SchnorrGroup>|
\end{itemize}


\textbf{\sffamily Constructors}
\begin{itemize}
\item \lstinline|protected| \lstinline|SchnorrNPServer|\lstinline|(NPServerContext<BigInteger, SchnorrGroup> ctx)| \\[-0.6em]




\end{itemize}


\subsubsection{Interface \lstinline|NPServerInterface|}
The common interface of NP servers based on different underlying group
 structures. \\
\noindent\begin{minipage}[t]{5cm}
\vspace{0.3em}
\hspace*{2em}
\begin{tikzpicture}
\umlclass[type=abstract]{NPServerInterface}{

}{
\umlvirt{+ getT() : int}
}
\end{tikzpicture}
\vspace{0.3em}
\end{minipage}



\textbf{\sffamily Superclasses and Interfaces}
\begin{itemize}
\item \lstinline|cryptocast.crypto.BroadcastSchemeKeyManager|
\item \lstinline|cryptocast.crypto.BroadcastSchemeUserManager|
\item \lstinline|cryptocast.crypto.Encryptor|
\end{itemize}



\textbf{\sffamily Methods}
\begin{itemize}
\item \lstinline|public int| \lstinline|getT|\lstinline|()|\\ \\[-0.6em]
\emph{Returns:} $t$



\end{itemize}

\subsubsection{Interface \lstinline|NPServerFactory|}
A strategy to create NP servers. \\
\noindent\begin{minipage}[t]{5cm}
\vspace{0.3em}
\hspace*{2em}
\begin{tikzpicture}
\umlclass[type=abstract]{NPServerFactory}{

}{
\umlvirt{+ construct(t : int) : NPServerInterface}
}
\end{tikzpicture}
\vspace{0.3em}
\end{minipage}





\textbf{\sffamily Methods}
\begin{itemize}
\item \lstinline|public NPServerInterface| \lstinline|construct|\lstinline|(int t)|\\ \\[-0.6em]
\emph{Returns:} A newly generated NP server instance
\begin{itemize}
\item \lstinline|t|: The parameter $t$ of the NP scheme
\end{itemize}



\end{itemize}


