\documentclass[a4paper,10pt]{scrartcl}
\usepackage[utf8]{inputenc}
\usepackage[T1]{fontenc}
\usepackage{booktabs}
\usepackage{xspace}
\usepackage{enumitem}
\usepackage{cite}
\usepackage{graphicx}
\usepackage{tikz}
\usetikzlibrary{arrows}
\usetikzlibrary{fit}
\usetikzlibrary{calc}
\usepackage{float}
\usepackage[section]{placeins} % don't move figures beyond the next section heading

% this is needed for forms and links within the text
\usepackage{hyperref}

% Variables
\newcommand{\authorName}{
   Mohammed~Abu~Jayyab,
   Niklas~Baumstark,
   Tobias~Gräf,
   Christoph~Michel,
   Amrei~Loose,
}
\newcommand{\authorNameEmph}{
   Mohammed~Abu~Jayyab,
   Niklas~Baumstark,
   \textbf{Tobias~Gräf},
   Christoph~Michel,
   Amrei~Loose,
}

\newcommand{\dateFirstVersion}{\today}
\newcommand{\customer}{Karlsruhe Institute of Technology}
\newcommand{\contractor}{A company}
\newcommand{\projectName}{Broadcast Encryption\xspace}

\newcommand{\doctitle}{\projectName (Design document)}
\title{\doctitle}
\author{\authorName}
\date{\today}

% less margin
\usepackage[margin=2.5cm]{geometry}

% horizontal line
\newcommand{\HRule}{\rule{\linewidth}{0.5mm}}

% more beautiful lists
\setlist{noitemsep}
\renewcommand{\labelitemi}{$\bullet$}
\renewcommand{\labelitemii}{$\diamond$}

% create a shorter version for tables
\newcommand\addrow[2]{#1 &#2\\ }
\newcommand\addheading[2]{\textbf{\sffamily #1} &\textbf{\sffamily #2}\\ \hline}
\newcommand\tabularhead{\begin{tabular}{lp{13cm}}
\hline
}

\newcommand\addmulrow[2]{ \begin{minipage}[t][][t]{2.5cm}#1\end{minipage}%
   &\begin{minipage}[t][][t]{8cm}
    \begin{enumerate} #2   \end{enumerate}
    \end{minipage}\\ }

\newenvironment{usecase}{\tabularhead}
{\hline\end{tabular}}

% a cross
\newcommand\X{$\times$}

% templates and default styles for figures and graphics
\tikzset{>=triangle 45}
\tikzset{font=\sffamily}

\newcommand{\tmpCaption}{}
\newenvironment{illustration}[1]
{
   \renewcommand{\tmpCaption}{#1}
   \begin{figure}[h!]
   \centering
}
{
   \caption{\tmpCaption}
   \end{figure}
}


\begin{document}

\maketitle
  \begin{tabular}[t]{ll}
	Projekt:       & \quad \projektName \\[1.2ex]
	Auftraggeber:  & \quad \auftraggeber\\[1.2ex]
	Auftragnehmer: & \quad \auftragnehmer\\[1.2ex]
  \end{tabular}

\begin{tabular}{|p{3 cm}|p{3 cm}|p{5 cm}|}
\hline
\textbf{Version} & \textbf{Datum} & \textbf{Autor(en)} \\
\hline
\hline
1.0 & 29.04.2012 & \authorName \\
\hline
\end{tabular}

\tableofcontents
\clearpage

\section{Introduction}
The software CryptoCast will provide a service for sending crypted data from a server to certain
amount of people.  This will be implemented as an unidirectional connection that makes it possible for the server
to send data without knowing how many people are receiving it and without the amount of traffic caused
by a bidirectional connection.  Also the type of the sent data will not be determined by the software. So it can be
used for all sorts of data exchange. For demonstration purposes we will implement a simple audio or video stream.

The server will be able to register new users and revoke specific persons if they are not allowed to receive the cast anymore. 
The client is an app on an Android smart phone that is used for receiving, decoding and displaying the data. 
The transport between them will be implemented with TCP but that can be replaced by another transport protocoll.



\section{Structure}
\subsection{Architecture}
\subsection{Class description}

\section{Sequences}

%\bibliography{../bibtex/references}{}
%\bibliographystyle{plain}

\end{document}
