\documentclass[a4paper,10pt]{scrartcl}
\usepackage[utf8]{inputenc}
\usepackage[T1]{fontenc}
\usepackage{booktabs}
\usepackage{xspace}
\usepackage{enumitem}
\usepackage{cite}
\usepackage{graphicx}
\usepackage{tikz}
\usetikzlibrary{arrows}
\usetikzlibrary{fit}
\usetikzlibrary{calc}
\usepackage{float}
\usepackage{listings}
\usepackage[section]{placeins} % don't move figures beyond the next section heading

% this is needed for forms and links within the text
\usepackage{hyperref}

% Variables
\newcommand{\authorName}{
   Mohammed~Abu~Jayyab,
   Niklas~Baumstark,
   Tobias~Gräf,
   Amrei~Loose,
   Christoph~Michel,
}
\newcommand{\authorNameEmph}{
   Mohammed~Abu~Jayyab,
   Niklas~Baumstark,
   \textbf{Tobias~Gräf},
   Christoph~Michel,
   Amrei~Loose,
}

\newcommand{\dateFirstVersion}{\today}
\newcommand{\customer}{Karlsruhe Institute of Technology}
\newcommand{\contractor}{A company}
\newcommand{\projectName}{Broadcast Encryption\xspace}

\newcommand{\doctitle}{\projectName (Design document)}
\title{\doctitle}
\author{\authorName}
\date{\today}

% less margin
\usepackage[margin=2.5cm]{geometry}

% horizontal line
\newcommand{\HRule}{\rule{\linewidth}{0.5mm}}

% more beautiful lists
\setlist{noitemsep}
\renewcommand{\labelitemi}{$\bullet$}
\renewcommand{\labelitemii}{$\diamond$}

% create a shorter version for tables
\newcommand\addrow[2]{#1 &#2\\ }
\newcommand\addheading[2]{\textbf{\sffamily #1} &\textbf{\sffamily #2}\\ \hline}
\newcommand\tabularhead{\begin{tabular}{lp{13cm}}
\hline
}

\newcommand\addmulrow[2]{ \begin{minipage}[t][][t]{2.5cm}#1\end{minipage}%
   &\begin{minipage}[t][][t]{8cm}
    \begin{enumerate} #2   \end{enumerate}
    \end{minipage}\\ }

\newenvironment{usecase}{\tabularhead}
{\hline\end{tabular}}

% a cross
\newcommand\X{$\times$}

% templates and default styles for figures and graphics
\tikzset{>=triangle 45}
\tikzset{font=\sffamily}

\newcommand{\tmpCaption}{}
\newenvironment{illustration}[1]
{
   \renewcommand{\tmpCaption}{#1}
   \begin{figure}[ht!]
   \centering
}
{
   \caption{\tmpCaption}
   \end{figure}
}

% code
\lstset{frame=tRBl, frameround=tttt, basicstyle=\footnotesize\ttfamily}
\lstset{numbers=left, numberstyle=\scriptsize}
\lstset{tabsize=4}
\lstset{language=Java}


\begin{document}

\maketitle
  \begin{tabular}[t]{ll}
	Projekt:       & \quad \projektName \\[1.2ex]
	Auftraggeber:  & \quad \auftraggeber\\[1.2ex]
	Auftragnehmer: & \quad \auftragnehmer\\[1.2ex]
  \end{tabular}

\begin{tabular}{|p{3 cm}|p{3 cm}|p{5 cm}|}
\hline
\textbf{Version} & \textbf{Datum} & \textbf{Autor(en)} \\
\hline
\hline
1.0 & 29.04.2012 & \authorName \\
\hline
\end{tabular}
\clearpage

\tableofcontents
\clearpage

\section{Introduction}
The software CryptoCast will provide a service for sending crypted data from a server to certain
amount of people.  This will be implemented as an unidirectional connection that makes it possible for the server
to send data without knowing how many people are receiving it and without the amount of traffic caused
by a bidirectional connection.  Also the type of the sent data will not be determined by the software. So it can be
used for all sorts of data exchange. For demonstration purposes we will implement a simple audio or video stream.

The server will be able to register new users and revoke specific persons if they are not allowed to receive the cast anymore. 
The client is an app on an Android smart phone that is used for receiving, decoding and displaying the data. 
The transport between them will be implemented with TCP but that can be replaced by another transport protocoll.

The implemented encryption algorithm will be the the one by Moni Naor and Benny Pinkas, but later every other fitting the
interface can be used. This is important because there might be another more effective algorithm 


\section{Structure}
\subsection{Architecture}

\section{API design}
\subsection{Package \lstinline!cryptocast.comm!}
\subsubsection{Class \lstinline|WritableByteChannelAdapter|}
Adapter to use a WritableByteChannel (for example, a socket instance) as an OutChannel. \\


\textbf{Superclasses and Interfaces}
\begin{itemize}
\item \lstinline|cryptocast.comm.OutChannel|
\end{itemize}



\textbf{Constructors}
\begin{itemize}
\item \lstinline|public| \lstinline|WritableByteChannelAdapter|\lstinline|(WritableByteChannel inner)|\\
Initializes the adapter
\begin{itemize}
\item \lstinline|inner|: The wrapped instance
\end{itemize}



\end{itemize}


\textbf{Methods}
\begin{itemize}
\item \lstinline|public void| \lstinline|send|\lstinline|(ByteBuffer data)|\\
Sends the given data.
\begin{itemize}
\item \lstinline|data|: the data to send
\end{itemize}



\end{itemize}

\subsubsection{Class \lstinline|SocketMulticastServer|}
This class implements channel-based communication via TCP. \\


\textbf{Superclasses and Interfaces}
\begin{itemize}
\item \lstinline|cryptocast.comm.OutChannel|
\end{itemize}



\textbf{Constructors}
\begin{itemize}
\item \lstinline|public| \lstinline|SocketMulticastServer|\lstinline|(ServerSocket socket)|\\
Creates a multicast server which uses the given socket.
\begin{itemize}
\item \lstinline|socket|: server socket
\end{itemize}



\end{itemize}


\textbf{Methods}
\begin{itemize}
\item \lstinline|public void| \lstinline|send|\lstinline|(ByteBuffer data)|\\
Sends bytes via the channel.
\begin{itemize}
\item \lstinline|data|: the data to send
\end{itemize}



\end{itemize}

\subsubsection{Class \lstinline|MessageInChannel|}
Wraps a byte-based InChannel and allows to use it as a message-based
 channel. \\




\textbf{Constructors}
\begin{itemize}
\item \lstinline|public| \lstinline|MessageInChannel|\lstinline|(InChannel inner)|\\
Creates a MessageInChannel which wraps the given inner channel.
\begin{itemize}
\item \lstinline|inner|: the wrapped channel
\end{itemize}



\end{itemize}


\textbf{Methods}
\begin{itemize}
\item \lstinline|public ByteBuffer| \lstinline|recvMessage|\lstinline|()|\\
Receives a message via the channel.

Returns: the received bytes

\end{itemize}

\subsubsection{Class \lstinline|MessageOutChannel|}
Wraps a byte-based OutChannel and allows to use it as a message-based
 channel. \\




\textbf{Constructors}
\begin{itemize}
\item \lstinline|public| \lstinline|MessageOutChannel|\lstinline|(OutChannel inner)|\\
Creates a new MessageOutChannel with the given OutChannel as inner channel.
\begin{itemize}
\item \lstinline|inner|: the OutChannel which will be wrapped
\end{itemize}



\end{itemize}


\textbf{Methods}
\begin{itemize}
\item \lstinline|public void| \lstinline|sendMessage|\lstinline|(ByteBuffer data)|\\
Sends the given message via the channel.
\begin{itemize}
\item \lstinline|data|: the data to send
\end{itemize}



\end{itemize}

\subsubsection{Class \lstinline|ReadableByteChannelAdapter|}
Adapter to use a ReadableByteChannel (for example, a socket instance) as an InChannel. \\


\textbf{Superclasses and Interfaces}
\begin{itemize}
\item \lstinline|cryptocast.comm.InChannel|
\end{itemize}



\textbf{Constructors}
\begin{itemize}
\item \lstinline|public| \lstinline|ReadableByteChannelAdapter|\lstinline|(ReadableByteChannel inner)|\\
Initializes the adapter
\begin{itemize}
\item \lstinline|inner|: The wrapped instance
\end{itemize}



\end{itemize}


\textbf{Methods}
\begin{itemize}
\item \lstinline|public ByteBuffer| \lstinline|recv|\lstinline|(int size)|\\
Receives data.
\begin{itemize}
\item \lstinline|size|: maximum amount of bytes to read
\end{itemize}

Returns: the received data

\end{itemize}

\subsubsection{Class \lstinline|MultiOutChannel|}
Multiplexes several OutChannels so that they can be used as a single
 destination. \\


\textbf{Superclasses and Interfaces}
\begin{itemize}
\item \lstinline|cryptocast.comm.OutChannel|
\end{itemize}



\textbf{Constructors}
\begin{itemize}
\item \lstinline|public| \lstinline|MultiOutChannel|\lstinline|()|




\end{itemize}


\textbf{Methods}
\begin{itemize}
\item \lstinline|public void| \lstinline|addChannel|\lstinline|(OutChannel channel)|\\
Adds the given channel to the list of receivers.
\begin{itemize}
\item \lstinline|channel|: the channel to add
\end{itemize}



\item \lstinline|public void| \lstinline|removeChannel|\lstinline|(OutChannel channel)|\\
Removes the given channel from the list of receivers.
\begin{itemize}
\item \lstinline|channel|: the channel to remove
\end{itemize}



\item \lstinline|public void| \lstinline|send|\lstinline|(ByteBuffer data)|




\end{itemize}

\subsubsection{Interface \lstinline|InChannel|}
None \\





\textbf{Methods}
\begin{itemize}
\item \lstinline|public ByteBuffer| \lstinline|recv|\lstinline|(int size)|\\
Receives data.
\begin{itemize}
\item \lstinline|size|: maximum amount of bytes to read
\end{itemize}

Returns: the received data

\end{itemize}

\subsubsection{Interface \lstinline|OutChannel|}
A byte-based communication channel where data can be sent to. \\





\textbf{Methods}
\begin{itemize}
\item \lstinline|public void| \lstinline|send|\lstinline|(ByteBuffer data)|\\
Sends the given data.
\begin{itemize}
\item \lstinline|data|: the data to send
\end{itemize}



\end{itemize}


\subsection{Package \lstinline!cryptocast.crypto.naorpinkas!}
\subsubsection{Class \lstinline|NaorPinkasClient|}
A client in the Naor-Pinkas broadcast encryption scheme. \\


\textbf{Superclasses and Interfaces}
\begin{itemize}
\item \lstinline|cryptocast.crypto.Decryptor<BigInteger>|
\end{itemize}



\textbf{Constructors}
\begin{itemize}
\item \lstinline|public| \lstinline|NaorPinkasClient|\lstinline|(NaorPinkasPersonalKey key)|\\
Initializes a Naor-Pinkas broadcast client
\begin{itemize}
\item \lstinline|key|: The personal key used to reconstruct a secret from the stream
\end{itemize}



\end{itemize}


\textbf{Methods}
\begin{itemize}
\item \lstinline|public BigInteger| \lstinline|decrypt|\lstinline|(ByteBuffer cipher)|\\
Decrypts a secret.
\begin{itemize}
\item \lstinline|cipher|: The encrypted secret
\end{itemize}

Returns: The decrypted secret

\end{itemize}

\subsubsection{Class \lstinline|NaorPinkasServer|}
A server in the Naor-Pinkas broadcast encryption scheme. This server is special in that it knows
 the entire polynomial and therefore all the private keys of its users. This makes the implementation
 of the interpolation algorithm a bit more efficient. \\


\textbf{Superclasses and Interfaces}
\begin{itemize}
\item \lstinline|cryptocast.crypto.BroadcastSchemeKeyManager<NaorPinkasIdentity>|
\item \lstinline|cryptocast.crypto.BroadcastSchemeUserManager<NaorPinkasIdentity>|
\item \lstinline|cryptocast.crypto.Encryptor<BigInteger>|
\item \lstinline|java.io.Serializable|
\end{itemize}



\textbf{Constructors}
\begin{itemize}
\item \lstinline|public| \lstinline|NaorPinkasServer|\lstinline|()|




\end{itemize}


\textbf{Methods}
\begin{itemize}
\item \lstinline|public ByteBuffer| \lstinline|encrypt|\lstinline|(BigInteger secret)|\\
Encrypts a secret
\begin{itemize}
\item \lstinline|secret|: the secret
\end{itemize}

Returns: The cipher text

\item \lstinline|public NaorPinkasIdentity| \lstinline|getIdentity|\lstinline|(int i)|\\
Returns: The identity with the given index
\begin{itemize}
\item \lstinline|i|: An index
\end{itemize}



\item \lstinline|public void| \lstinline|revoke|\lstinline|(NaorPinkasIdentity x)|\\
Revokes a user



\item \lstinline|public boolean| \lstinline|isRevoked|\lstinline|(NaorPinkasIdentity id)|\\
Returns: whether the user is revoked
\begin{itemize}
\item \lstinline|id|: The identity of the user
\end{itemize}



\item \lstinline|public NaorPinkasPersonalKey| \lstinline|getPersonalKey|\lstinline|(NaorPinkasIdentity id)|\\
Returns: The private key of the user
\begin{itemize}
\item \lstinline|id|: The identity to look up
\end{itemize}



\end{itemize}

\subsubsection{Class \lstinline|NaorPinkasPersonalKey|}
A user's personal key in the Naor-Pinkas broadcast encryption scheme. \\


\textbf{Superclasses and Interfaces}
\begin{itemize}
\item \lstinline|java.security.PrivateKey|
\end{itemize}



\textbf{Constructors}
\begin{itemize}
\item \lstinline|public| \lstinline|NaorPinkasPersonalKey|\lstinline|()|




\end{itemize}


\textbf{Methods}
\begin{itemize}
\item \lstinline|public String| \lstinline|getAlgorithm|\lstinline|()|




\item \lstinline|public byte[]| \lstinline|getEncoded|\lstinline|()|




\item \lstinline|public String| \lstinline|getFormat|\lstinline|()|




\end{itemize}

\subsubsection{Class \lstinline|NaorPinkasShare|}
A share in the Naor-Pinkas broadcast encryption scheme. It consists of a tuple
 $(r, I, g^{r P(I)})$. $t + 1$ distinct shares of this form are sufficient to restore the
 value $g^{r P(0)}$, where $t$ is the degree of the polynomial $P$. \\


\textbf{Superclasses and Interfaces}
\begin{itemize}
\item \lstinline|cryptocast.crypto.Share<BigInteger>|
\end{itemize}



\textbf{Constructors}
\begin{itemize}
\item \lstinline|public| \lstinline|NaorPinkasShare|\lstinline|()|




\end{itemize}


\textbf{Methods}
\begin{itemize}
\item \lstinline|public boolean| \lstinline|isComplete|\lstinline|()|\\
Returns: whether the share is complete and can be used to restore
 the value.



\item \lstinline|public byte[]| \lstinline|pack|\lstinline|()|\\
Returns: A binary representation of the share



\item \lstinline|public BigInteger| \lstinline|restore|\lstinline|()|\\
Restore the value represented by this share.

Returns: The restored value

\end{itemize}

\subsubsection{Class \lstinline|NaorPinkasIdentity|}
An identity in the Naor-Pinkas broadcast encryption scheme \\




\textbf{Constructors}
\begin{itemize}
\item \lstinline|private| \lstinline|NaorPinkasIdentity|\lstinline|(BigInteger id)|




\end{itemize}


\textbf{Methods}
\begin{itemize}
\item \lstinline|private BigInteger| \lstinline|getId|\lstinline|()|




\end{itemize}


\subsection{Package \lstinline!cryptocast.client!}
\subsubsection{Class \lstinline|StreamViewerActivity|}
This activity is responsible for decrypting the received data
 stream and viewing it. \\




\textbf{Constructors}
\begin{itemize}
\item \lstinline|public| \lstinline|StreamViewerActivity|\lstinline|()|




\end{itemize}


\textbf{Methods}
\begin{itemize}
\item \lstinline|public void| \lstinline|processStream|\lstinline|(InputStream inputStream)|\\
Processes the input stream of data
\begin{itemize}
\item \lstinline|inputStream|: the data stream
\end{itemize}



\item \lstinline|public void| \lstinline|saveStream|\lstinline|(InputStream inputStream)|\\
Saves the data stream
\begin{itemize}
\item \lstinline|inputStream|: the data stream
\end{itemize}



\end{itemize}

\subsubsection{Class \lstinline|ServerHistory|}
This class is responsible for saving recently selected servers
 and their corresponding key files. \\




\textbf{Constructors}
\begin{itemize}
\item \lstinline|public| \lstinline|ServerHistory|\lstinline|()|\\
Constructor for this class



\end{itemize}


\textbf{Methods}
\begin{itemize}
\item \lstinline|public Map<String, File>| \lstinline|getServers|\lstinline|()|\\
Returns: the servers



\item \lstinline|public void| \lstinline|setServers|\lstinline|(Map<String, File> servers)|

\begin{itemize}
\item \lstinline|servers|: the servers to set
\end{itemize}



\end{itemize}

\subsubsection{Class \lstinline|ServerLoginActivity|}
This class represents the activity to connect to the server.
 Before connecting this activity start the {@link KeyChoiceActivity} to 
 let the user choose an encryption key file. When the client receives a 
 data stream the {@link StreamViewerActivity} is started to process the 
 stream and show its contents. \\


\textbf{Superclasses and Interfaces}
\begin{itemize}
\item \lstinline|cryptocast.comm.InChannel|
\end{itemize}



\textbf{Constructors}
\begin{itemize}
\item \lstinline|public| \lstinline|ServerLoginActivity|\lstinline|()|




\end{itemize}


\textbf{Methods}
\begin{itemize}
\item \lstinline|public void| \lstinline|connectToServer|\lstinline|(String serverAddress)|\\
Connects to server
\begin{itemize}
\item \lstinline|serverAddress|: the server address
\end{itemize}



\item \lstinline|private void| \lstinline|checkValidAddress|\lstinline|(String address)|\\
Checks if server address is valid
\begin{itemize}
\item \lstinline|address|: the server address
\end{itemize}



\item \lstinline|public void| \lstinline|chooseKeyFile|\lstinline|()|\\
Shows activity to choose key file



\item \lstinline|public void| \lstinline|saveServer|\lstinline|(String serverAddress, File file)|\\
Saves servers and their corresponding key files.
\begin{itemize}
\item \lstinline|serverAddress|: the server address
\item \lstinline|file|: the key file
\end{itemize}



\item \lstinline|public void| \lstinline|showStream|\lstinline|()|\\
Shows activity to view downloaded stream



\item \lstinline|public void| \lstinline|processError|\lstinline|()|\\
Processes any errors



\item \lstinline|public ByteBuffer| \lstinline|recv|\lstinline|(int size)|\\
Receives data from the channel. It is decrypted on the fly.
\begin{itemize}
\item \lstinline|size|: the amount of bytes to receive
\end{itemize}



\end{itemize}

\subsubsection{Class \lstinline|Traffics|}
This class is for showing the received traffics. \\




\textbf{Constructors}
\begin{itemize}
\item \lstinline|public| \lstinline|Traffics|\lstinline|()|\\
Constructor for this class



\end{itemize}


\textbf{Methods}
\begin{itemize}
\item \lstinline|public long| \lstinline|getReceivedTraffics|\lstinline|()|\\
Returns: the receivedTraffics



\item \lstinline|public void| \lstinline|setReceivedTraffics|\lstinline|(long receivedTraffics)|

\begin{itemize}
\item \lstinline|receivedTraffics|: the receivedTraffics to set
\end{itemize}



\end{itemize}

\subsubsection{Class \lstinline|KeyChoiceActivity|}
This activity lets a user choose an encryption key file 
 which is then sent to the server for authentication. \\




\textbf{Constructors}
\begin{itemize}
\item \lstinline|public| \lstinline|KeyChoiceActivity|\lstinline|()|




\end{itemize}


\textbf{Methods}
\begin{itemize}
\item \lstinline|public void| \lstinline|chooseKeyFile|\lstinline|(File file)|\\
Chooses a key file
\begin{itemize}
\item \lstinline|file|: the key file
\end{itemize}



\end{itemize}


\subsection{Package \lstinline!cryptocast.util!}
\subsubsection{Class \lstinline|CommandLineInterface|}
A simple framework for command line programs. \\




\textbf{Constructors}
\begin{itemize}
\item \lstinline|public| \lstinline|CommandLineInterface|\lstinline|(InputStream in, PrintStream out, PrintStream err)|\\
Initializes a new CLI instance
\begin{itemize}
\item \lstinline|in|: The stream for program input
\item \lstinline|out|: The stream for program output
\item \lstinline|err|: The stream for error output
\end{itemize}



\end{itemize}


\textbf{Methods}
\begin{itemize}
\item \lstinline|public int| \lstinline|run|\lstinline|(String[] args)|\\
Runs the application.
\begin{itemize}
\item \lstinline|args|: The command line arguments
\end{itemize}

Returns: The exit code

\item \lstinline|protected abstract void| \lstinline|start|\lstinline|(String[] args)|\\
The main program logic (must be overridden by subclasses)
\begin{itemize}
\item \lstinline|args|: The command line arguments
\end{itemize}



\item \lstinline|protected void| \lstinline|printf|\lstinline|(String format, Object[] args...)|\\
Prints a string to the output stream
\begin{itemize}
\item \lstinline|format|: The string to print (printf format string)
\item \lstinline|args|: The printf arguments
\end{itemize}



\item \lstinline|protected void| \lstinline|print|\lstinline|(String str)|\\
Prints a string to the output stream
\begin{itemize}
\item \lstinline|str|: The string to print
\end{itemize}



\item \lstinline|protected void| \lstinline|println|\lstinline|(String str)|\\
Prints a string to the output stream after appending a newline
\begin{itemize}
\item \lstinline|str|: The string to print
\end{itemize}



\item \lstinline|protected String| \lstinline|getErrorFormat|\lstinline|()|\\
Returns: the string format to use for writing error messages to the
 screen.



\item \lstinline|protected void| \lstinline|fatalError|\lstinline|(String format, Object[] args...)|\\
Prints an error and exits
\begin{itemize}
\item \lstinline|format|: The error message (printf format string)
\item \lstinline|args|: The printf arguments
\end{itemize}



\item \lstinline|protected void| \lstinline|exit|\lstinline|(int status)|\\
Exits the application
\begin{itemize}
\item \lstinline|status|: The exit code
\end{itemize}



\item \lstinline|protected void| \lstinline|usage|\lstinline|()|\\
Prints usage information



\item \lstinline|protected String| \lstinline|getBasicUsage|\lstinline|()|\\
Returns: basic usage information for the program (should be overridden)



\item \lstinline|protected void| \lstinline|printAdditionalUsage|\lstinline|()|\\
Prints additional usage information (may be overridden)



\end{itemize}

\subsubsection{Class \lstinline|CommandLineInterface.Exit|}
Signals the exit of the application. \\


\textbf{Superclasses and Interfaces}
\begin{itemize}
\item \lstinline|java.lang.Throwable|
\end{itemize}



\textbf{Constructors}
\begin{itemize}
\item \lstinline|public| \lstinline|CommandLineInterface.Exit|\lstinline|(int status)|\\
Initializes a new Exit instance
\begin{itemize}
\item \lstinline|status|: The exit code
\end{itemize}



\end{itemize}


\textbf{Methods}
\begin{itemize}
\item \lstinline|public int| \lstinline|getStatus|\lstinline|()|\\
Returns: the exit code



\end{itemize}

\subsubsection{Class \lstinline|InteractiveCommandLineInterface|}
A framework class to implement interactive command-line interfaces. The class implements
 a read-parse-execute main loop and provides hooks for subclasses to implement the missing
 functionality. \\


\textbf{Superclasses and Interfaces}
\begin{itemize}
\item \lstinline|cryptocast.util.CommandLineInterface|
\end{itemize}



\textbf{Constructors}
\begin{itemize}
\item \lstinline|public| \lstinline|InteractiveCommandLineInterface|\lstinline|(InputStream in, PrintStream out, PrintStream err)|\\
Initializes a new interactive CLI instance
\begin{itemize}
\item \lstinline|in|: The stream for program input
\item \lstinline|out|: The stream for program output
\item \lstinline|err|: The stream for error output
\end{itemize}



\end{itemize}


\textbf{Methods}
\begin{itemize}
\item \lstinline|protected void| \lstinline|start|\lstinline|(String[] args)|\\
The main program logic. This method just starts the main loop.
\begin{itemize}
\item \lstinline|args|: The command line arguments (ignored by default)
\end{itemize}



\item \lstinline|protected void| \lstinline|mainloop|\lstinline|()|\\
Starts the interactive Prompt-Read-Evaluate main loop.



\item \lstinline|protected abstract void| \lstinline|performCommand|\lstinline|(String cmd, String[] args)|\\
Executes the given command with the given arguments. Must be implemented by subclasses.
\begin{itemize}
\item \lstinline|cmd|: The command name
\item \lstinline|args|: The command arguments
\end{itemize}



\item \lstinline|protected void| \lstinline|error|\lstinline|(String format, Object[] args...)|\\
Helper function to trigger an error withing a command's execution and break out to
 the main loop.
\begin{itemize}
\item \lstinline|format|: The format string
\item \lstinline|args|: The format string arguments
\end{itemize}



\item \lstinline|protected String| \lstinline|getPrompt|\lstinline|()|\\
Returns: The input prompt



\end{itemize}

\subsubsection{Class \lstinline|InteractiveCommandLineInterface.CommandError|}
An error within one of the commands. Will be caught by the main loop \\


\textbf{Superclasses and Interfaces}
\begin{itemize}
\item \lstinline|java.lang.Throwable|
\end{itemize}



\textbf{Constructors}
\begin{itemize}
\item \lstinline|public| \lstinline|InteractiveCommandLineInterface.CommandError|\lstinline|(String msg)|\\
Initializes the error
\begin{itemize}
\item \lstinline|msg|: The error message
\end{itemize}



\end{itemize}


\textbf{Methods}
\begin{itemize}
\item \lstinline|public String| \lstinline|getMessage|\lstinline|()|\\
Returns: The associated error message



\end{itemize}


\subsection{Package \lstinline!cryptocast.crypto!}
\subsubsection{Class \lstinline|Share.InsufficientInformationException|}
An exception that is thrown when a secret cannot be restored because
 of missing information. \\


\textbf{Superclasses and Interfaces}
\begin{itemize}
\item \lstinline|java.lang.Exception|
\end{itemize}



\textbf{Constructors}
\begin{itemize}
\item \lstinline|public| \lstinline|Share.InsufficientInformationException|\lstinline|()|




\end{itemize}


\subsubsection{Class \lstinline|LagrangeInterpolation<T>|}
Performs a lagrange interpolation of a polynomial \\


\begin{itemize}
\item \lstinline|<T>|: The type of items of the polynomial's field
\end{itemize}


\textbf{Constructors}
\begin{itemize}
\item \lstinline|public| \lstinline|LagrangeInterpolation|\lstinline|(Polynomial<T> poly)|\\
Initializes the algorithm
\begin{itemize}
\item \lstinline|poly|: The polynomial to interpolate
\end{itemize}



\end{itemize}


\textbf{Methods}
\begin{itemize}
\item \lstinline|public T[]| \lstinline|computeCoefficients|\lstinline|()|\\
Returns: The lagrange coefficients of the associated polynomial



\end{itemize}

\subsubsection{Class \lstinline|IntegersModuloPrime|}
The field $\mathbb{Z}/p\mathbb{Z}$ of integers modulo a prime $p$ \\


\textbf{Superclasses and Interfaces}
\begin{itemize}
\item \lstinline|cryptocast.crypto.Field<BigInteger>|
\end{itemize}



\textbf{Constructors}
\begin{itemize}
\item \lstinline|public| \lstinline|IntegersModuloPrime|\lstinline|(BigInteger p)|\\
Initializes the field
\begin{itemize}
\item \lstinline|p|: A prime number
\end{itemize}



\end{itemize}


\textbf{Methods}
\begin{itemize}
\item \lstinline|public BigInteger| \lstinline|add|\lstinline|(BigInteger a, BigInteger b)|




\item \lstinline|public BigInteger| \lstinline|multiply|\lstinline|(BigInteger a, BigInteger b)|




\item \lstinline|public BigInteger| \lstinline|negate|\lstinline|(BigInteger a)|




\item \lstinline|public BigInteger| \lstinline|invert|\lstinline|(BigInteger a)|




\item \lstinline|public BigInteger| \lstinline|zero|\lstinline|()|




\item \lstinline|public BigInteger| \lstinline|one|\lstinline|()|




\item \lstinline|public BigInteger| \lstinline|randomElement|\lstinline|()|




\end{itemize}

\subsubsection{Class \lstinline|Polynomial<T>|}
A polynomial $P$ over a field \\


\begin{itemize}
\item \lstinline|<T>|: The type of the field's elements
\end{itemize}


\textbf{Constructors}
\begin{itemize}
\item \lstinline|public| \lstinline|Polynomial|\lstinline|(Field<T> field, T[] coefficients)|\\
Initializes a polynomial
\begin{itemize}
\item \lstinline|field|: An instance of the field over which the polynomial is formed
\item \lstinline|coefficients|: The coefficients $c_i$ of the polynomial ($0 \leq i \leq n$).
 The polynomial is defined as $P(x) := \sum_{i=0}^n c_i x^i = c_0 + c_1
 x + ... + c_n x^n$
\end{itemize}



\end{itemize}


\textbf{Methods}
\begin{itemize}
\item \lstinline|public Field<T>| \lstinline|getField|\lstinline|()|\\
Returns: The field associated with this polynomial



\item \lstinline|public T| \lstinline|evaluate|\lstinline|(T x)|\\
Evaluates the polynomial at a single point x.
\begin{itemize}
\item \lstinline|x|: The
\end{itemize}

Returns: P(x)

\item \lstinline|public T[]| \lstinline|evaluateMulti|\lstinline|(T[] xs)|\\
Evaluates the polynomial at multiple points in time complexity $\Theta(n\cdot\log
 n)$ where $n$ is the degree of the polynomial
\begin{itemize}
\item \lstinline|xs|: The points $x_i$ to evaluate
\end{itemize}

Returns: The array a defined by $a_i := P(x_i)$

\item \lstinline|public T| \lstinline|getCoefficient|\lstinline|(int i)|\\
Returns: $c_i$
\begin{itemize}
\item \lstinline|i|: The index of the coefficient to get ($0 \leq i \leq n$), where
          $n$ is the degree of the polynomial
\end{itemize}



\item \lstinline|public int| \lstinline|getDegree|\lstinline|()|\\
Returns: The degree of the polynomial



\item \lstinline|public static Polynomial<T>| \lstinline|createRandomPolynomial|\lstinline|(Field<T> field, int degree)|\\
Generates a random polynomial over the field
\begin{itemize}
\item \lstinline|field|: An instance of the field over which the polynomial is formed
\item \lstinline|degree|: The degree of the generated polynomial
\end{itemize}

Returns: The generated polynomial

\end{itemize}

\subsubsection{Class \lstinline|Field<T>|}
Represents a field over values of type T \\


\begin{itemize}
\item \lstinline|<T>|: The values we work on
\end{itemize}


\textbf{Constructors}
\begin{itemize}
\item \lstinline|public| \lstinline|Field|\lstinline|()|




\end{itemize}


\textbf{Methods}
\begin{itemize}
\item \lstinline|public abstract T| \lstinline|add|\lstinline|(T a, T b)|\\
Adds two elements of the field
\begin{itemize}
\item \lstinline|a|: first element
\item \lstinline|b|: second element
\end{itemize}

Returns: The value $a + b$

\item \lstinline|public abstract T| \lstinline|multiply|\lstinline|(T a, T b)|\\
Multiplies two elements of the field
\begin{itemize}
\item \lstinline|a|: first element
\item \lstinline|b|: second element
\end{itemize}

Returns: The value $a \cdot b$

\item \lstinline|public abstract T| \lstinline|negate|\lstinline|(T a)|\\
Returns: The additive inverse $-a$ of $a$
\begin{itemize}
\item \lstinline|a|: An element of the field
\end{itemize}



\item \lstinline|public abstract T| \lstinline|invert|\lstinline|(T a)|\\
Returns: The multiplicative inverse $a^{-1}$ of $a$
\begin{itemize}
\item \lstinline|a|: An element of the field
\end{itemize}



\item \lstinline|public abstract T| \lstinline|zero|\lstinline|()|\\
Returns: The zero element of the field



\item \lstinline|public abstract T| \lstinline|one|\lstinline|()|\\
Returns: The one element of the field



\item \lstinline|public abstract T| \lstinline|randomElement|\lstinline|()|\\
Returns: A random element of the field



\item \lstinline|public T| \lstinline|subtract|\lstinline|(T a, T b)|\\
Subtracts two elements of the field
\begin{itemize}
\item \lstinline|a|: first element
\item \lstinline|b|: second element
\end{itemize}

Returns: The value $a - b$

\item \lstinline|public T| \lstinline|divide|\lstinline|(T a, T b)|\\
Divides two elements of the field
\begin{itemize}
\item \lstinline|a|: first element
\item \lstinline|b|: second element
\end{itemize}

Returns: The value $\frac{a}{b}$

\item \lstinline|public T| \lstinline|pow|\lstinline|(T a, int e)|\\
Raises an element of the field to an integer power
\begin{itemize}
\item \lstinline|a|: The element of the field
\item \lstinline|e|: The exponent
\end{itemize}

Returns: The value $a^e$

\end{itemize}

\subsubsection{Class \lstinline|BroadcastEncryptionServer<ID>|}
The server side of a broadcast encryption scheme. \\


\textbf{Superclasses and Interfaces}
\begin{itemize}
\item \lstinline|cryptocast.comm.OutChannel|
\item \lstinline|java.lang.Runnable|
\end{itemize}

\begin{itemize}
\item \lstinline|<ID>|: The type of the identities
\end{itemize}


\textbf{Constructors}
\begin{itemize}
\item \lstinline|public| \lstinline|BroadcastEncryptionServer|\lstinline|(MessageOutChannel inner, BroadcastSchemeUserManager<ID> context, Encryptor<BigInteger> enc)|\\
Initializes a broadcast encryption server.
\begin{itemize}
\item \lstinline|inner|: The message-based communication channel to send outgoing data to
\item \lstinline|context|: The user management context
\item \lstinline|enc|: The encryption context
\end{itemize}



\end{itemize}


\textbf{Methods}
\begin{itemize}
\item \lstinline|public void| \lstinline|run|\lstinline|()|\\
Run the worker that handles periodic group key broadcasts and sends
 queued data packages.



\item \lstinline|public void| \lstinline|send|\lstinline|(ByteBuffer data)|\\
Send plaintext data to the channel. It will be encryted and broadcasted
 on the fly.
\begin{itemize}
\item \lstinline|data|: The data to send
\end{itemize}



\item \lstinline|public void| \lstinline|revoke|\lstinline|(ID id)|\\
Revoke a user.
\begin{itemize}
\item \lstinline|id|: The identity of the user
\end{itemize}



\end{itemize}

\subsubsection{Class \lstinline|BroadcastEncryptionClient|}
The client side of a broadcast encryption scheme. \\


\textbf{Superclasses and Interfaces}
\begin{itemize}
\item \lstinline|cryptocast.comm.InChannel|
\end{itemize}



\textbf{Constructors}
\begin{itemize}
\item \lstinline|public| \lstinline|BroadcastEncryptionClient|\lstinline|(MessageInChannel inner, Decryptor<BigInteger> dec)|\\
Initializes a broadcast encryption client.
\begin{itemize}
\item \lstinline|inner|: The message-based underlying communication channel.
\item \lstinline|dec|: The decryption context
\end{itemize}



\end{itemize}


\textbf{Methods}
\begin{itemize}
\item \lstinline|public ByteBuffer| \lstinline|recv|\lstinline|(int size)|\\
Receive data from the channel. It is decrypted on the fly.
\begin{itemize}
\item \lstinline|size|: amount of bytes to receive
\end{itemize}



\end{itemize}

\subsubsection{Interface \lstinline|Decryptor<S>|}
A strategy to decrypt a single secret \\


\begin{itemize}
\item \lstinline|<S>|: the type of the secret
\end{itemize}



\textbf{Methods}
\begin{itemize}
\item \lstinline|public S| \lstinline|decrypt|\lstinline|(ByteBuffer cipher)|\\
Decrypts a secret.
\begin{itemize}
\item \lstinline|cipher|: The encrypted secret
\end{itemize}

Returns: The decrypted secret

\end{itemize}

\subsubsection{Interface \lstinline|Share<S>|}
Represents a possibly incomplete piece of information to restore a secret value. \\


\begin{itemize}
\item \lstinline|<S>|: the type of secret.
\end{itemize}



\textbf{Methods}
\begin{itemize}
\item \lstinline|public boolean| \lstinline|isComplete|\lstinline|()|\\
Returns: whether the share is complete and can be used to restore
 the value.



\item \lstinline|public S| \lstinline|restore|\lstinline|()|\\
Restore the value represented by this share.

Returns: The restored value

\item \lstinline|public byte[]| \lstinline|pack|\lstinline|()|\\
Returns: A binary representation of the share



\end{itemize}

\subsubsection{Interface \lstinline|BroadcastSchemeUserManager<ID>|}
Manages a set of user identites \\


\begin{itemize}
\item \lstinline|<ID>|: The type of the identities
\end{itemize}



\textbf{Methods}
\begin{itemize}
\item \lstinline|public ID| \lstinline|getIdentity|\lstinline|(int i)|\\
Returns: The identity with the given index
\begin{itemize}
\item \lstinline|i|: An index
\end{itemize}



\item \lstinline|public void| \lstinline|revoke|\lstinline|(ID id)|\\
Revokes a user
\begin{itemize}
\item \lstinline|id|: The identity of the user
\end{itemize}



\item \lstinline|public boolean| \lstinline|isRevoked|\lstinline|(ID id)|\\
Returns: whether the user is revoked
\begin{itemize}
\item \lstinline|id|: The identity of the user
\end{itemize}



\end{itemize}

\subsubsection{Interface \lstinline|Encryptor<S>|}
A strategy to encrypt a single secret \\


\begin{itemize}
\item \lstinline|<S>|: the type of the secret
\end{itemize}



\textbf{Methods}
\begin{itemize}
\item \lstinline|public ByteBuffer| \lstinline|encrypt|\lstinline|(S secret)|\\
Encrypts a secret
\begin{itemize}
\item \lstinline|secret|: the secret
\end{itemize}

Returns: The cipher text

\end{itemize}

\subsubsection{Interface \lstinline|BroadcastSchemeKeyManager<ID>|}
Manages the private keys of a set of users. \\


\begin{itemize}
\item \lstinline|<ID>|: The type of the user identities
\end{itemize}



\textbf{Methods}
\begin{itemize}
\item \lstinline|public PrivateKey| \lstinline|getPersonalKey|\lstinline|(ID id)|\\
Returns: The private key of the user
\begin{itemize}
\item \lstinline|id|: The identity to look up
\end{itemize}



\end{itemize}

\subsubsection{Interface \lstinline|ShareCombinator<S, T>|}
Implements a strategy to merge multiple Shares into a single share with
 more information. \\


\begin{itemize}
\item \lstinline|<S>|: The type of the secret
\item \lstinline|<T>|: The type of the shares
\end{itemize}



\textbf{Methods}
\begin{itemize}
\item \lstinline|public T| \lstinline|combine|\lstinline|(T a, T b)|\\
Combines two shares.
\begin{itemize}
\item \lstinline|a|: The first share
\item \lstinline|b|: The second share
\end{itemize}

Returns: a new share containing the information from both a and b.

\end{itemize}


\subsection{Package \lstinline!cryptocast.server!}
\subsubsection{Class \lstinline|Command|}
This class represents a command and saves all legal commands. \\




\textbf{Constructors}
\begin{itemize}
\item \lstinline|private| \lstinline|Command|\lstinline|(String cmd, String shortDescription, String longDescription)|\\
Creates a new Command with the given values
\begin{itemize}
\item \lstinline|cmd|: None
\end{itemize}



\end{itemize}


\subsubsection{Class \lstinline|Controller<ID>|}
Deals with user-interactions and therefore changes data in Model if necessary. \\


\begin{itemize}
\item \lstinline|<ID>|: The type of the user identities
\end{itemize}


\textbf{Constructors}
\begin{itemize}
\item \lstinline|public| \lstinline|Controller|\lstinline|(ServerData data, Shell shell, BroadcastEncryptionServer<ID> encServer)|\\
Initializes a new controller with the given arguments.
\begin{itemize}
\item \lstinline|data|: The data administrated by this controller.
\item \lstinline|shell|: The operator interface from which this controller gets its input.
\item \lstinline|encServer|: The server to which the data is send.
\end{itemize}



\end{itemize}


\textbf{Methods}
\begin{itemize}
\item \lstinline|public void| \lstinline|init|\lstinline|()|\\
Initializes the server on start by loading data from a file.



\item \lstinline|public void| \lstinline|keyGen|\lstinline|(int amtRevocable, int amtPrivateKeys, File keyDir)|\\
Tries to start a new group to which data can be sent by generating private keys.
\begin{itemize}
\item \lstinline|amtRevocable|: The amount of user which can be revoked.
\item \lstinline|amtPrivateKeys|: The amount of private keys which are produced.
\item \lstinline|keyDir|: The directory to save the keyfiles in.
\end{itemize}



\item \lstinline|public void| \lstinline|addUser|\lstinline|(User user)|\\
Adds a new user and assigns a private key to that user.
\begin{itemize}
\item \lstinline|user|: The user that is added.
\end{itemize}



\item \lstinline|public void| \lstinline|revokeUser|\lstinline|(User user)|\\
Bans a user from the stream by adding it to the list of revoked users.
\begin{itemize}
\item \lstinline|user|: The user that is revoked.
\end{itemize}



\item \lstinline|public void| \lstinline|authorizeUser|\lstinline|(User user)|\\
Authorizes a user to watch the stream by removing it from the list of revoked users.
\begin{itemize}
\item \lstinline|user|: The user that is unbanned.
\end{itemize}



\item \lstinline|public void| \lstinline|stream|\lstinline|(File data)|\\
Starts the data stream
\begin{itemize}
\item \lstinline|data|: The file from which the data is read
\end{itemize}



\item \lstinline|public void| \lstinline|showStatistics|\lstinline|()|\\
Prints information about traffic



\item \lstinline|public void| \lstinline|showUsers|\lstinline|()|\\
Prints users and the keys assigned to them.



\item \lstinline|public void| \lstinline|showInfo|\lstinline|()|\\
Prints information about the data which is currently sent.



\end{itemize}

\subsubsection{Class \lstinline|User<ID>|}
This Class represents an User. \\


\begin{itemize}
\item \lstinline|<ID>|: The type of the user identities
\end{itemize}


\textbf{Constructors}
\begin{itemize}
\item \lstinline|public| \lstinline|User|\lstinline|(String name, ID id)|\\
Creates a User with the given name.
\begin{itemize}
\item \lstinline|name|: name of the user
\end{itemize}



\end{itemize}


\textbf{Methods}
\begin{itemize}
\item \lstinline|public String| \lstinline|getName|\lstinline|()|\\
Returns: the name of this user.



\item \lstinline|public ID| \lstinline|getIdentity|\lstinline|()|\\
Returns: the id of this user.



\end{itemize}

\subsubsection{Class \lstinline|Shell<ID>|}
Gets the arguments from the command line and deals with illegal input. \\


\textbf{Superclasses and Interfaces}
\begin{itemize}
\item \lstinline|cryptocast.util.InteractiveCommandLineInterface|
\end{itemize}

\begin{itemize}
\item \lstinline|<ID>|: The type of the user identities.
\end{itemize}


\textbf{Constructors}
\begin{itemize}
\item \lstinline|public| \lstinline|Shell|\lstinline|(InputStream in, PrintStream out, PrintStream err)|\\
Creates a new Shell object with the given parameters.
\begin{itemize}
\item \lstinline|in|: The input stream
\item \lstinline|out|: Stream to write normal output to.
\item \lstinline|err|: Stream to write error messages to.
\end{itemize}



\end{itemize}


\textbf{Methods}
\begin{itemize}
\item \lstinline|protected void| \lstinline|performCommand|\lstinline|(String cmd, String[] args)|




\item \lstinline|private void| \lstinline|help|\lstinline|()|\\
Prints all commands this shell can perform with information about how to use them.



\end{itemize}

\subsubsection{Class \lstinline|ServerData<ID>|}
Contains the data which is changed by Controller and presented on CLI. \\


\textbf{Superclasses and Interfaces}
\begin{itemize}
\item \lstinline|java.io.Serializable|
\end{itemize}

\begin{itemize}
\item \lstinline|<ID>|: The type of the user identities
\end{itemize}


\textbf{Constructors}
\begin{itemize}
\item \lstinline|public| \lstinline|ServerData|\lstinline|()|




\end{itemize}


\textbf{Methods}
\begin{itemize}
\item \lstinline|public <any>| \lstinline|createNewUser|\lstinline|(String name)|\\
Creates and saves a new user by name.
\begin{itemize}
\item \lstinline|name|: The user's name
\end{itemize}



\item \lstinline|public <any>| \lstinline|getUserByName|\lstinline|(String name)|\\
Retrieves a user by name
\begin{itemize}
\item \lstinline|name|: The user's name
\end{itemize}

Returns: A user instance, if it was found, or absent otherwise

\item \lstinline|public <any>| \lstinline|getPersonalKey|\lstinline|(User user)|\\
Retrieves a user's personal key
\begin{itemize}
\item \lstinline|user|: The user object
\end{itemize}

Returns: The private key

\item \lstinline|public <any>| \lstinline|getStreamDir|\lstinline|()|\\
Returns: Returns the directory from which is currently streamed.



\item \lstinline|public <any>| \lstinline|setStreamDir|\lstinline|(File streamDir)|\\
Saves the directory from which is streamed.
\begin{itemize}
\item \lstinline|streamDir|: The directory from which is streamed.
\end{itemize}



\end{itemize}

\subsubsection{Class \lstinline|Main|}
The main method to start the server \\




\textbf{Constructors}
\begin{itemize}
\item \lstinline|public| \lstinline|Main|\lstinline|()|




\end{itemize}


\textbf{Methods}
\begin{itemize}
\item \lstinline|public static void| \lstinline|main|\lstinline|(String[] args)|

\begin{itemize}
\item \lstinline|args|: command line arguments
\end{itemize}



\end{itemize}




\section{Sequences}

%\bibliography{../bibtex/references}{}
%\bibliographystyle{plain}

\end{document}
