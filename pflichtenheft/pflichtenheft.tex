\documentclass[a4paper,10pt]{article}
\usepackage{amssymb} % needed for math
\usepackage{amsmath} % needed for math
\usepackage[utf8]{inputenc} % this is needed for german umlauts
\usepackage[ngerman]{babel} % this is needed for german umlauts
\usepackage[T1]{fontenc}    % this is needed for correct output of umlauts in pdf
\usepackage[margin=2.5cm]{geometry} %layout
\usepackage{booktabs}

% this is needed for forms and links within the text
\usepackage{hyperref}

% glossary, see http://en.wikibooks.org/wiki/LaTeX/Glossary
% has to be loaded AFTER hyperref so that entries are clickable
\usepackage[nonumberlist]{glossaries}

% The following is needed in order to make the code compatible
% with both latex/dvips and pdflatex.
\ifx\pdftexversion\undefined
\usepackage[dvips]{graphicx}
\else
\usepackage[pdftex]{graphicx}
\DeclareGraphicsRule{*}{mps}{*}{}
\fi

\makeglossary

% Variables
\newcommand{\authorName}{Mohammed Abu~Jayyab, Niklas Baumstark,
                         Tobias Gräf, Amrei Loose, Christoph Michel}
\newcommand{\dateFirstVersion}{\today}
\newcommand{\customer}{Karlsruhe Institute of Technology}
\newcommand{\contractor}{Eine Firma}
\newcommand{\projectName}{Broadcast-Verschlüsselung}
\newcommand{\tags}{\authorName, Pflichtenheft, KIT, Informatik,
                   PSE, Broadcast-Verschlüsselung}
\newcommand{\glossarName}{Glossar}
\newcommand{\doctitle}{\projectName\ (Pflichtenheft)}
\title{\doctitle}
\author{\authorName}
\date{\today}

% PDF Meta information
\hypersetup{
  pdfauthor   = {\authorName},
  pdfkeywords = {\tags},
  pdftitle    = {\doctitle}
}

% Create a shorter version for tables.
\newcommand\addrow[2]{#1 &#2\\ }
\newcommand\addheading[2]{#1 &#2\\ \hline}
\newcommand\tabularhead{\begin{tabular}{lp{13cm}}
\hline
}

\newcommand\addmulrow[2]{ \begin{minipage}[t][][t]{2.5cm}#1\end{minipage}%
   &\begin{minipage}[t][][t]{8cm}
    \begin{enumerate} #2   \end{enumerate}
    \end{minipage}\\ }

\newenvironment{usecase}{\tabularhead}
{\hline\end{tabular}}

\begin{document}

\maketitle
  \begin{tabular}[t]{ll}
	Projekt:       & \quad \projektName \\[1.2ex]
	Auftraggeber:  & \quad \auftraggeber\\[1.2ex]
	Auftragnehmer: & \quad \auftragnehmer\\[1.2ex]
  \end{tabular}

\begin{tabular}{|p{3 cm}|p{3 cm}|p{5 cm}|}
\hline
\textbf{Version} & \textbf{Datum} & \textbf{Autor(en)} \\
\hline
\hline
1.0 & 29.04.2012 & \authorName \\
\hline
\end{tabular}
\clearpage

\tableofcontents

\clearpage
\section{Einleitung}
\section{Zielbestimmung}
Die Software soll es ermöglichen Inhalte verschlüsselt von einem Server an verschiedene Klienten verteilen. Entgegen dem aktuellen Stand der Technik skaliert das System mit den ausgeschlossenen Nutzern anstelle der Anzahl der Benutzer.

\subsection{Musskriterien}
% Musskriterien
\newcommand{\mussKuerzel}{MU}
\begin{usecase}
\addheading{Nummer}{Beschreibung}
\addrow{/\mussKuerzel10/} {Der Server muss Daten versenden können.}
\addrow{/\mussKuerzel20/} {Der Server muss Daten verschlüsseln können.}
\addrow{/\mussKuerzel25/} {Der Server muss über eine Trafficanzeige bzw. Log verfügen.}
\addrow{/\mussKuerzel30/} {Der Server muss Schlüssel generieren können.}
\addrow{/\mussKuerzel40/} {Der Server muss jedem Benutzer einen Schlüssel zuordnen können.}
\addrow{/\mussKuerzel50/} {Der Server muss Nutzer ausschließen können.}
\addrow{/\mussKuerzel60/} {Der Server muss ausgeschlossene Nutzer speichern.}
\addrow{/\mussKuerzel70/} {Der Client muss sich zu einem Server verbinden können.}
\addrow{/\mussKuerzel80/} {Der Client muss Daten empfangen können.}
\addrow{/\mussKuerzel90/} {Der Client muss Daten entschlüsseln können.}
\addrow{/\mussKuerzel100/} {Der Client muss entschlüsselte Daten darstellen können.}
\end{usecase}

% Wunschkriterien
\subsection{Wunschkriterien}
\newcommand{\wunschKuerzel}{WU}
\begin{usecase}
\addheading{Nummer}{Beschreibung}
\addrow{/\wunschKuerzel10/} {Der Client kann mehrere Serverfavoriten speichern, welche über ein Kontextmenü auswählbar sind.}
\addrow{/\wunschKuerzel20/} {Der Client kann in einen "WIFI only" Modus wechseln, sodass er nur Daten im WLAN empfängt.}
\addrow{/\wunschKuerzel30/} {Der Client verfügt über eine Anzeige des angefallenen Traffics.}
\addrow{/\wunschKuerzel40/} {Der Client kann empfangene Daten Buffern, um eine flüssigere Wiedergabe von empfangnen Daten zu ermöglichen.}
\end{usecase}

% Wunschkriterien
\subsection{Abgrenzungskriterien}
\newcommand{\abgrenzungKuerzel}{AG}
\begin{usecase}
\addheading{Nummer}{Beschreibung}
\addrow{/\abgrenzungKuerzel10/} {Es wird nicht definiert, welche Art von Daten übertragen werden.}
\end{usecase}

% Produkteinsatz
\section{Produkteinsatz}
Das Produkt wird benutzt um Daten an exklusive Benutzer zu senden.
\subsection{Anwendungsbereich}
Das Produkt kann in allen Bereichen zum Einsatz kommen, die das Verteilen von Daten an bestimmte Nutzergruppen erfordert.
\subsection{Zielgruppen}

\subsection{Betriebsbedingungen}

% Produktfunktion
\section{Produktfunktion}

% Produktdaten
\section{Produktdaten}
\newcommand{\datenKuerzel}{PD}
\begin{usecase}
\addheading{Nummer}{Beschreibung}
\addrow{/\datenKuerzel10/} {Der Server speichert ausgeschlossene Nutzer.}
\addrow{/\datenKuerzel20/} {Der Server ordnet Nutzern einen Schlüssel zu.}
\addrow{/\datenKuerzel30/} {Der Server speichert zur Verschlüsselung erforderliche Daten.}
\end{usecase}

% Produktumgebung
\section{Produktumgebung}
Der Server muss auf einem Java-fähigen System laufen.
Der Client muss auf einem portablen Android Gerät laufen.
\subsection{Software}

\subsection{Hardware}
Android-Tablet/-Smartphone, Werte?!
% Systemmodell
\section{Systemmodell}

% Produktleistung
\section{Produktleistung}

% GUI
\section{GUI}

% Testfälle
\section{Testfälle}

%Entwicklungsumgebung
\section{Entwicklungsumgebung}

\clearpage
\newglossaryentry{broadcastenc}
{
  name=Broadcast-Verschl"usselung,
  description={Ein Verschl"usselungsverfahren f"ur unidirektionale Streams, bei dem der
  Sender die Untermenge der Empf"anger bestimmen kann, die in der Lage ist, den Stream
  zu entschl"usseln}
}
\newglossaryentry{sessionkey}
{
  name=Session-Key,
  description={Der symmetrische Schl"ussel, mit dem die Nutzdaten verschl"usselt werden.
  Jeder nicht ausgeschlossene Client muss diesen Schl"ussel aus den Server-Nachrichten
  berechnen k"onnen}
}
\newglossaryentry{client}
{
  name=Client,
  description={Ein Client (deutsch: Kunde) ist eine Software, die Daten oder Anwendungen
  von einem Server anfordert.}
}
\newglossaryentry{server}
{
  name=Server,
  description={Ein Server (deutsch: Diener) ist in der Informatik ein Dienstleister,
	der in einem Computersystem Daten oder Ressourcen zur Verfügung stellt.}
}
\newglossaryentry{payload}
{
  name=Payload,
  description={Payload (deutsch: Nutzdaten) sind während einer Kommunikation zwischen
  zwei Partnern transportierten Daten eines Datenpakets, die keine Steuer- oder Protokollinformationen enthalten.
  Nutzdaten sind unter anderem Sprache, Text, Zeichen, Bilder und Töne.}
}
\newglossaryentry{traffic}
{
  name=Traffic,
  description={Traffic (deutsch: Daten-Verkehr) bezeichnet durch Abrufe von Webdokumenten
  und anderen Dateien einer Website entstehende Datentransfer-Volumina zwischen einem Server und dem Client-Programm.}
}
\newglossaryentry{key}
{
  name=Schlüssel,
  description={Ein Schlüssel in der Kryptologie ist zusätzliche Information die man benötigt um eine
	Nachricht zu chiffrieren bzw. dechiffrieren. Normalerweise besteht ein Schlüssel aus einer Folge von
	Zahlen oder Buchstaben die nur dem Absender und dem Empfänger einer Nachricht bekannt sind.}
}
\newglossaryentry{hostname}
{
  name=Hostname,
  description={Der Hostname (auch Sitename) ist die eindeutige Bezeichnung eines Rechners in einem Netzwerk,
  mit der er im Netzwerk angesprochen wird.}
}
\newglossaryentry{gui}
{
  name=GUI,
  description={Graphical user interface(GUI) (deutsch : grafische Benutzeroberfläche) ist eine Software-Komponente,
  die einem Computerbenutzer die Interaktion mit der Maschine über grafische Symbole erlaubt.}
}

% Setze den richtigen Namen für das Glossar
\renewcommand*{\glossaryname}{\section{\glossarName}}

% Drucke das gesamte Glossar
\glsaddall
\printglossaries


\end{document}
