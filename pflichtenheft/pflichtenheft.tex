\documentclass[a4paper,10pt]{article}
\usepackage{amssymb} % needed for math
\usepackage{amsmath} % needed for math
\usepackage[utf8]{inputenc} % this is needed for german umlauts
\usepackage[ngerman]{babel} % this is needed for german umlauts
\usepackage[T1]{fontenc}    % this is needed for correct output of umlauts in pdf
\usepackage[margin=2.5cm]{geometry} %layout
\usepackage{booktabs}
\usepackage{xspace}
\usepackage{enumitem}
\usepackage{cite}

% this is needed for forms and links within the text
\usepackage{hyperref}

% glossary, see http://en.wikibooks.org/wiki/LaTeX/Glossary
% has to be loaded AFTER hyperref so that entries are clickable
\usepackage[nonumberlist]{glossaries}

% The following is needed in order to make the code compatible
% with both latex/dvips and pdflatex.
\ifx\pdftexversion\undefined
\usepackage[dvips]{graphicx}
\else
\usepackage[pdftex]{graphicx}
\DeclareGraphicsRule{*}{mps}{*}{}
\fi

\makeglossary

% Variables
\newcommand{\authorName}{
      Mohammed Abu~Jayyab, Niklas Baumstark,
      Tobias Gräf, Amrei Loose, \textbf{Christoph Michel}}
\newcommand{\dateFirstVersion}{\today}
\newcommand{\customer}{Karlsruhe Institute of Technology}
\newcommand{\contractor}{Eine Firma}
\newcommand{\projectName}{Broadcast-Verschlüsselung\xspace}
\newcommand{\tags}{\authorName, Pflichtenheft, KIT, Informatik,
                   PSE, Broadcast-Verschlüsselung}
\newcommand{\glossarName}{Glossar}
\newcommand{\doctitle}{\projectName (Pflichtenheft)}
\title{\doctitle}
\author{\authorName}
\date{\today}

% PDF Meta information
\hypersetup{
  pdfauthor   = {\authorName},
  pdfkeywords = {\tags},
  pdftitle    = {\doctitle}
}

% less margin
\usepackage[margin=2.5cm]{geometry}

% horizontal line
\newcommand{\HRule}{\rule{\linewidth}{0.5mm}}

% more beautiful lists
\setlist{noitemsep}
\renewcommand{\labelitemi}{$\bullet$}
\renewcommand{\labelitemii}{$\diamond$}

% create a shorter version for tables
\newcommand\addrow[2]{#1 &#2\\ }
\newcommand\addheading[2]{\textbf{\sffamily #1} &\textbf{\sffamily #2}\\ \hline}
\newcommand\tabularhead{\begin{tabular}{lp{13cm}}
\hline
}

\newcommand\addmulrow[2]{ \begin{minipage}[t][][t]{2.5cm}#1\end{minipage}%
   &\begin{minipage}[t][][t]{8cm}
    \begin{enumerate} #2   \end{enumerate}
    \end{minipage}\\ }

\newenvironment{usecase}{\tabularhead}
{\hline\end{tabular}}

% a cross
\newcommand\X{$\times$}

% templates and default styles for figures and graphics
\tikzset{>=triangle 45}
\tikzset{font=\sffamily}

\newcommand{\tmpCaption}{}
\newenvironment{illustration}[1]
{
   \renewcommand{\tmpCaption}{#1}
   \begin{figure}[h!]
   \centering
}
{
   \caption{\tmpCaption}
   \end{figure}
}


\begin{document}

\maketitle
  \begin{tabular}[t]{ll}
	Projekt:       & \quad \projektName \\[1.2ex]
	Auftraggeber:  & \quad \auftraggeber\\[1.2ex]
	Auftragnehmer: & \quad \auftragnehmer\\[1.2ex]
  \end{tabular}

\begin{tabular}{|p{3 cm}|p{3 cm}|p{5 cm}|}
\hline
\textbf{Version} & \textbf{Datum} & \textbf{Autor(en)} \\
\hline
\hline
1.0 & 29.04.2012 & \authorName \\
\hline
\end{tabular}

\tableofcontents
\clearpage

\section{Einleitung}

Heutzutage gängige Streamingdienste im Internet basieren bis auf wenige Ausnahmen auf
einem Client-Server-Modell, bei dem jeder Client eine eigene Verbindung zum Server
aufbaut, um einen Stream zu empfangen. Das Trafficaufkommen, das durch diese Art der
Kommunikation verursacht wird sowie die notwendige Bandbreite sind enorm. Nur große
Anbieter von Inhalten können sich diese Form der Verteilung überhaupt leisten.

Mit der naheliegenden Erkenntnis, dass bei dem beschriebenen Verfahren dieselben Daten
vielfach an verschiedene Empfänger versendet werden, ergibt sich ein alternatives
Vorgehen auf der Basis von Multicast: Inhalte werden vom Server nur einmal versendet
und über das Internet an mehrere Empfänger zugestellt. Damit wird vor allem der Sender,
aber auch die gesamte Internet-Infrastruktur entlastet.

Ein großes Problem stellt allerdings die Zugangskontrolle für diese Multicast-Streams
dar: Da nun der Server nicht mehr weiß, wer eigentlich den Stream empfängt, kann
er auch nicht verhindern, dass unauthorisierte Benutzer Zugang erhalten. Die Lösung
dieses Problems erfordert den Einsatz von speziellen Verschlüsselungsverfahren,
sodass zwar jeder den Stream empfangen, aber nur authorisierte Benutzer den Stream
entschlüsseln können. Besonderes Augenmerk muss dabei auf Effizienz gelegt werden,
da der Kommunikationsaufwand durch die Verschlüsselung nicht wesentlich erhöht werden
darf.

Wir wurden daher beauftragt, einen Prototyp zu entwickeln, der ein
Broadcast"=Verschlüsselungsverfahren mit speziellen wünschenswerten Eigenschaften
demonstriert.

\section{Zielbestimmung}

Die entwickelte Software ist eine Client/Server-Kombination, die es ermöglicht,
Inhalte verschlüsselt von einem Server an verschiedene Clients verteilen. Dafür
soll kein bidirektionales Kommunikationsmedium erforderlich sein.

Es wird ein Verschlüsselungsverfahren eingesetzt, welches die besondere
Eigenschaft besitzt, nicht mit der Gesamtzahl der Empfänger, sondern mit der Anzahl
ausgeschlossener Benutzer zu skalieren.

\subsection{Musskriterien}

\begin{itemize}

\item Server
\begin{itemize}
   \item sendet Daten an eine Empfängergruppe. Zu Demonstrationszwecken wird
         ein einfacher Audio- oder Videostream als Payload eingesetzt.
   \item erlaubt es, nicht mehr authorisierte Benutzer auszuschließen.
\end{itemize}

\item Client
\begin{itemize}
   \item erlaubt es, sich mit einem Server zu verbinden (einer Gruppe beizutreten).
   \item empfängt Daten vom Server und stellt sie dar.
   \item merkt sich zuletzt ausgew"ahlte Server, sodass ein erneuter Zugriff schnell
         m"oglich ist.
\end{itemize}

\item Kommunikation und Verschl"usselung
\begin{itemize}
   \item Nur ein unidirektionaler Kommunikationskanal vom Server zum Client wird
         vorausgesetzt. Zu Demonstrationszwecken wird TCP als Transportprotokoll
         verwendet, aber auch zuverl"assige Multicastprotokolle k"amen als Medium
         infrage.
   \item Daten werden verschl"usselt "ubertragen. Der Server bietet entsprechende
         Optionen zur Erzeugung von Schl"usseln und zur Kontrolle und Konfiguration
         der Verschl"usselungsschicht. Jeder Benutzer erh"alt "uber einen sicheren
         Kanal einen privaten Schl"ussel (z.B. eine Datei), mit dem er seine
         Clientsoftware konfigurieren kann.
   \item Das verwendete Broadcast"=Verschl"usselungsverfahren basiert auf den
         in~\cite[Section~2.2]{Naor00} oder~\cite{Garg10} beschriebenen Verfahren
         oder einer Variante der Algorithmen mit vergleichbaren kryptografischen
         Eigenschaften.
   \item Die Verschl"usselung erfordert im laufenden Betrieb nur einen
         Kommunikationsoverhead, der sublinear in der Gesamtanzahl der Benutzer und
         stattdessen linear oder quasilinear in der Anzahl der ausgeschlossenen
         Benutzer ist.
\end{itemize}
\end{itemize}

\subsection{Wunschkriterien}

\begin{itemize}

\item Server
\begin{itemize}
   \item f"uhrt Statistik "uber Nutzdatenmenge und Traffic, um den Kommunikationsaufwand
         der Verschl"usselung zu analysieren.
\end{itemize}

\item Client
\begin{itemize}
   \item verf"ugt "uber eine Anzeige des angefallenen Traffics.
   \item puffert oder speichert Inhalte, sodass z.B. in einem Stream auch
         zur"uckgespult werden kann.
\end{itemize}

\item Kommunikation und Verschl"usselung
\begin{itemize}
   \item Ein effizienteres Verfahren als das in~\cite{Naor00} beschriebene wird
         entwickelt und implementiert.
\end{itemize}
\end{itemize}

\subsection{Abgrenzungskriterien}
\begin{itemize}
   \item Es wird kein Framework zur Verf"ugung gestellt, welches verschiedene
         Streamtypen implementiert. Stattdessen bleibt die Implementierung
         unabh"angig von den zugrundeliegenden Nutzdaten und ist so flexibel,
         dass eine konkrete Implementierung eines neuen Streamtyps mit
         problemlos m"oglich ist.
   \item Es wird kein Multicast-Protokoll implementiert. Stattdessen wird
         zur Demonstrationszwecken TCP zur Kommunikation benutzt und die Software
         ist so flexibel, dass eine Adaptation an ein zuverl"assiges
         Multicastverfahren problemlos m"oglich ist.
\end{itemize}

\section{Produkteinsatz}
Das Produkt wird benutzt um Inhalte an eine festgelegte Benutzergruppe zu verteilen.
\subsection{Anwendungsbereich}
Das Produkt kann in allen Bereichen zum Einsatz kommen, die das Verteilen von Daten an
bestimmte Nutzergruppen erfordert.
\subsection{Zielgruppen}

\subsection{Betriebsbedingungen}

\section{Produktfunktion}

\subsection{Grundfunktionen}

\newcommand{\mussKuerzel}{MU}
\begin{usecase}
\addheading{Nummer}{Beschreibung}
\addrow{/\mussKuerzel10/} {Der Server kann Daten versenden.}
\addrow{/\mussKuerzel20/} {Der Server kann Daten verschlüsseln.}
\addrow{/\mussKuerzel30/} {Der Server kann Schlüssel generieren.}
\addrow{/\mussKuerzel40/} {Der Server kann jedem Benutzer einen Schlüssel zuordnen.}
\addrow{/\mussKuerzel50/} {Der Server kann Nutzer ausschließen.}
\addrow{/\mussKuerzel60/} {Der Client kann sich zu einem Server verbinden.}
\addrow{/\mussKuerzel70/} {Der Client kann Daten empfangen.}
\addrow{/\mussKuerzel80/} {Der Client kann Daten entschlüsseln.}
\addrow{/\mussKuerzel90/} {Der Client stellt entschlüsselte Daten dar.}
\addrow{/\mussKuerzel100/} {Der Client merkt sich die letzten ausgewählten Server,
                            sodass ein erneuter Zugriff schnell möglich ist.}
\addrow{/\mussKuerzel110/} {Die Verschlüsselung erfolgt mit dem vorgegeben Verfahren.}
\end{usecase}

\subsection{Erweiterte Funktionen}
\newcommand{\wunschKuerzel}{WU}
\begin{usecase}
\addheading{Nummer}{Beschreibung}
\addrow{/\wunschKuerzel10/} {Der Benutzer kann im Client mehrere Serverfavoriten
                             mit Namen verwalten, welche dann über ein Kontextmenü
                             auswählbar sind.}
\addrow{/\wunschKuerzel20/} {Der Server führt Statistiken über Nutzdatenmenge und Traffic,
                             um den Kommunikationsaufwand für die Verschlüsselung zu
                             analysieren}
\addrow{/\wunschKuerzel30/} {Der Client verfügt über eine Anzeige des angefallenen Traffics.}
\addrow{/\wunschKuerzel40/} {Der Client puffert empfangene Daten, um eine flüssigere
                             Wiedergabe zu ermöglichen.}
\addrow{/\wunschKuerzel50/} {Ein effizienteres Verschlüsselungsverfahren als das vorgebene
                             wird entwickelt und implementiert.}
\addrow{/\wunschKuerzel60/} {Mehrere Quellen können Daten an dieselbe Multicast-Gruppe senden.}
\end{usecase}

\section{Produktdaten}
\newcommand{\datenKuerzel}{PD}
\begin{usecase}
\addheading{Nummer}{Beschreibung}
\addrow{/\datenKuerzel10/} {Der Server speichert ausgeschlossene Nutzer.}
\addrow{/\datenKuerzel20/} {Der Server ordnet Nutzern einen Schlüssel zu.}
\addrow{/\datenKuerzel30/} {Der Server speichert zur Verschlüsselung erforderliche Daten.}
\addrow{/\datenKuerzel40/} {Daten die Performance betreffend werden auf dem Server gespeichert.}
\end{usecase}

\section{Produktumgebung}
Der Server muss auf einem Java-fähigen System laufen.
Der Client muss auf einem portablen Gerät auf Basis des Android-Betriebssystems laufen.
\subsection{Software}

\subsection{Hardware}
Android-Tablet/-Smartphone, Werte?!

\section{Systemmodell}

% TODO nur Anregungen was hier stehen soll!
\section{Produktleistung}
* Angaben über die mögliche Anzahl ausgeschlossener Nutzer
* Dauer des Groupkeyupdates, da in dieser Zeit nichts empfangen werden kann
* Machbarkeit der Keygeneration

\section{GUI}

\section{Testfälle}

\section{Entwicklungsumgebung}

\clearpage
\newglossaryentry{broadcastenc}
{
  name=Broadcast-Verschl"usselung,
  description={Ein Verschl"usselungsverfahren f"ur unidirektionale Streams, bei dem der
  Sender die Untermenge der Empf"anger bestimmen kann, die in der Lage ist, den Stream
  zu entschl"usseln}
}
\newglossaryentry{sessionkey}
{
  name=Session-Key,
  description={Der symmetrische Schl"ussel, mit dem bei der Broadcast"=Verschl"usselung die
  Nutzdaten verschl"usselt werden. Jeder nicht ausgeschlossene Client muss diesen Schl"ussel
  aus den Server-Nachrichten berechnen k"onnen}
}
\newglossaryentry{server}
{
  name=Server,
  description={Eine Instanz, der in einem Computersystem Daten oder Ressourcen zur Verfügung stellt}
}
\newglossaryentry{client}
{
  name=Client,
  description={Eine Instanz, die Daten oder Anwendungen von einem Server anfordert}
}
\newglossaryentry{traffic}
{
  name=Traffic,
  description={Durch Netzwerkübertragungen entstehender Datenfluss}
}
\newglossaryentry{key}
{
  name=Schlüssel,
  description={Ein Schlüssel in der Kryptologie ist zusätzliche Information, die man benötigt um eine
	Nachricht zu chiffrieren bzw. dechiffrieren. Normalerweise besteht ein Schlüssel aus einer Folge von
	Zahlen oder Buchstaben, die entweder nur dem Empf"anger, oder sowohl Absender als auch Empfänger
        einer Nachricht bekannt sind}
}
\newglossaryentry{hostname}
{
  name=Hostname,
  description={Die eindeutige Bezeichnung, mit der ein Rechner er im Netzwerk angesprochen wird}
}
\newglossaryentry{gui}
{
  name=GUI (Graphical User Interface),
  description={Eine Software-Komponente, die einem Computerbenutzer die Interaktion mit der Maschine
   über grafische Symbole erlaubt}
}

% Setze den richtigen Namen für das Glossar
\renewcommand*{\glossaryname}{\section{\glossarName}}

% Drucke das gesamte Glossar
\glsaddall
\printglossaries



\bibliography{../bibtex/references}{}
\bibliographystyle{plain}

\end{document}
