\documentclass[a4paper,10pt]{article}
\usepackage{amssymb} % needed for math
\usepackage{amsmath} % needed for math
\usepackage[utf8]{inputenc} % this is needed for german umlauts
\usepackage[ngerman]{babel} % this is needed for german umlauts
\usepackage[T1]{fontenc}    % this is needed for correct output of umlauts in pdf
\usepackage[margin=2.5cm]{geometry} %layout
\usepackage{booktabs}

% this is needed for forms and links within the text
\usepackage{hyperref}

% glossary, see http://en.wikibooks.org/wiki/LaTeX/Glossary
% has to be loaded AFTER hyperref so that entries are clickable
\usepackage[nonumberlist]{glossaries}

% The following is needed in order to make the code compatible
% with both latex/dvips and pdflatex.
\ifx\pdftexversion\undefined
\usepackage[dvips]{graphicx}
\else
\usepackage[pdftex]{graphicx}
\DeclareGraphicsRule{*}{mps}{*}{}
\fi

\makeglossary

% Variables
\newcommand{\authorName}{Mohammed Abu~Jayyab, Niklas Baumstark,
                         Tobias Gräf, Amrei Loose, Christoph Michel}
\newcommand{\dateFirstVersion}{\today}
\newcommand{\customer}{Karlsruhe Institute of Technology}
\newcommand{\contractor}{Eine Firma}
\newcommand{\projectName}{Broadcast-Verschlüsselung}
\newcommand{\tags}{\authorName, Pflichtenheft, KIT, Informatik,
                   PSE, Broadcast-Verschlüsselung}
\newcommand{\glossarName}{Glossar}
\newcommand{\doctitle}{\projectName\ (Pflichtenheft)}
\title{\doctitle}
\author{\authorName}
\date{\today}

% PDF Meta information
\hypersetup{
  pdfauthor   = {\authorName},
  pdfkeywords = {\tags},
  pdftitle    = {\doctitle}
}

% Create a shorter version for tables.
\newcommand\addrow[2]{#1 &#2\\ }
\newcommand\addheading[2]{#1 &#2\\ \hline}
\newcommand\tabularhead{\begin{tabular}{lp{13cm}}
\hline
}

\newcommand\addmulrow[2]{ \begin{minipage}[t][][t]{2.5cm}#1\end{minipage}%
   &\begin{minipage}[t][][t]{8cm}
    \begin{enumerate} #2   \end{enumerate}
    \end{minipage}\\ }

\newenvironment{usecase}{\tabularhead}
{\hline\end{tabular}}

\begin{document}

\maketitle
  \begin{tabular}[t]{ll}
	Projekt:       & \quad \projektName \\[1.2ex]
	Auftraggeber:  & \quad \auftraggeber\\[1.2ex]
	Auftragnehmer: & \quad \auftragnehmer\\[1.2ex]
  \end{tabular}

\begin{tabular}{|p{3 cm}|p{3 cm}|p{5 cm}|}
\hline
\textbf{Version} & \textbf{Datum} & \textbf{Autor(en)} \\
\hline
\hline
1.0 & 29.04.2012 & \authorName \\
\hline
\end{tabular}

\tableofcontents

\clearpage
\section{Einleitung}

Heutzutage gängige Streamingdienste im Internet basieren bis auf wenige Ausnahmen auf
einem Client-Server-Modell, bei dem jeder Client eine eigene Verbindung zum Server
aufbaut, um einen Stream zu empfangen. Das Trafficaufkommen, das durch diese Art der
Kommunikation verursacht wird sowie die notwendige Bandbreite sind enorm. Nur große
Anbieter von Inhalten können sich diese Form der Verteilung überhaupt leisten.

Mit der naheliegenden Erkenntnis, dass bei dem beschriebenen Verfahren dieselben Daten
vielfach an verschiedene Empfänger versendet werden, ergibt sich ein alternatives
Vorgehen auf der Basis von Multicast: Inhalte werden vom Server nur einmal versendet
und über das Internet an mehrere Empfänger zugestellt. Damit wird vor allem der Sender,
aber auch die gesamte Internet-Infrastruktur entlastet.

Ein großes Problem stellt allerdings die Zugangskontrolle für diese Multicast-Streams
dar: Da nun der Server nicht mehr weiß, wer eigentlich den Stream empfängt, kann
er auch nicht verhindern, dass unauthorisierte Benutzer Zugang erhalten. Die Lösung
dieses Problems erfordert den Einsatz von speziellen Verschlüsselungsverfahren,
sodass zwar jeder den Stream empfangen, aber nur authorisierte Benutzer den Stream
entschlüsseln können. Besonderes Augenmerk muss dabei auf Effizienz gelegt werden,
da der Kommunikationsaufwand durch die Verschlüsselung nicht wesentlich erhöht werden
darf.

Wir wurden daher beauftragt, einen Prototyp zu entwickeln, der ein effizientes
Broadcast-Verschlüsselungsverfahren mit speziellen wünschenswerten Eigenschaften
demonstriert.

\section{Zielbestimmung}
Die Software soll es ermöglichen, Inhalte verschlüsselt von einem Server an verschiedene Klienten verteilen. Die Kommunikation soll dabei ausschließlich unidirektional erfolgen.
Entgegen dem aktuellen Stand der Technik skaliert das System mit der Anzahl ausgeschlossenen Nutzern anstelle der gesamten Anzahl der Benutzer.

\subsection{Musskriterien}
\newcommand{\mussKuerzel}{MU}
\begin{usecase}
\addheading{Nummer}{Beschreibung}
\addrow{/\mussKuerzel10/} {Der Server kann Daten versenden.}
\addrow{/\mussKuerzel20/} {Der Server kann Daten verschlüsseln.}
\addrow{/\mussKuerzel30/} {Der Server kann Schlüssel generieren.}
\addrow{/\mussKuerzel40/} {Der Server kann jedem Benutzer einen Schlüssel zuordnen.}
\addrow{/\mussKuerzel50/} {Der Server kann Nutzer ausschließen.}
\addrow{/\mussKuerzel60/} {Der Client kann sich zu einem Server verbinden.}
\addrow{/\mussKuerzel70/} {Der Client kann Daten empfangen.}
\addrow{/\mussKuerzel80/} {Der Client kann Daten entschlüsseln.}
\addrow{/\mussKuerzel90/} {Der Client stellt entschlüsselte Daten dar.}
\addrow{/\mussKuerzel100/} {Der Client merkt sich die letzten ausgewählten Server,
                            sodass ein erneuter Zugriff schnell möglich ist.}
\addrow{/\mussKuerzel110/} {Die Verschlüsselung erfolgt mit dem vorgegeben Verfahren.}
\end{usecase}

%TODO 'zu implementierendes Verschlüsselungsverfahren' spezifizieren
\subsection{Wunschkriterien}
\newcommand{\wunschKuerzel}{WU}
\begin{usecase}
\addheading{Nummer}{Beschreibung}
\addrow{/\wunschKuerzel10/} {Der Benutzer kann im Client mehrere Serverfavoriten
                             mit Namen verwalten, welche dann über ein Kontextmenü
                             auswählbar sind.}
\addrow{/\wunschKuerzel20/} {Der Server führt Statistiken über Nutzdatenmenge und Traffic,
                             um den Kommunikationsaufwand für die Verschküsselung zu
                             analysieren}
\addrow{/\wunschKuerzel30/} {Der Client verfügt über eine Anzeige des angefallenen Traffics.}
\addrow{/\wunschKuerzel40/} {Der Client puffert empfangene Daten, um eine flüssigere
                             Wiedergabe zu ermöglichen.}
\addrow{/\wunschKuerzel50/} {Ein effizienteres Verschlüsselungsverfahren als das vorgebene
                             wird entwickelt und implementiert.}
\addrow{/\wunschKuerzel60/} {Mehrere Quellen können Daten an dieselbe Multicast-Gruppe senden.}
\end{usecase}

\subsection{Abgrenzungskriterien}
\newcommand{\abgrenzungKuerzel}{AG}
\begin{usecase}
\addheading{Nummer}{Beschreibung}
\addrow{/\abgrenzungKuerzel10/} {Es wird nicht definiert, welche Art von Daten
                                 übertragen werden.}
\addrow{/\abgrenzungKuerzel20/} {Keine bidirektionale Kommunikation zwischen Server
                                 und Clients.}
\addrow{/\abgrenzungKuerzel30/} {Es wird kein Multicast-Protokoll implementiert. Stattdessen
                                 wird TCP zur Kommunikation eingesetzt und die Software
                                 ist so flexibel erweiterbar, dass eine Adaption auf
                                 ein zuverlässiges Multicastverfahren leicht möglich ist}
\end{usecase}

\section{Produkteinsatz}
Das Produkt wird benutzt um Inhalte an eine festgelegte Benutzergruppe zu verteilen.
\subsection{Anwendungsbereich}
Das Produkt kann in allen Bereichen zum Einsatz kommen, die das Verteilen von Daten an
bestimmte Nutzergruppen erfordert.
\subsection{Zielgruppen}

\subsection{Betriebsbedingungen}

\section{Produktfunktion}

\section{Produktdaten}
\newcommand{\datenKuerzel}{PD}
\begin{usecase}
\addheading{Nummer}{Beschreibung}
\addrow{/\datenKuerzel10/} {Der Server speichert ausgeschlossene Nutzer.}
\addrow{/\datenKuerzel20/} {Der Server ordnet Nutzern einen Schlüssel zu.}
\addrow{/\datenKuerzel30/} {Der Server speichert zur Verschlüsselung erforderliche Daten.}
\addrow{/\datenKuerzel40/} {Daten die Performance betreffend werden auf dem Server gespeichert.}
\end{usecase}

\section{Produktumgebung}
Der Server muss auf einem Java-fähigen System laufen.
Der Client muss auf einem portablen Gerät auf Basis des Android-Betriebssystems laufen.
\subsection{Software}

\subsection{Hardware}
Android-Tablet/-Smartphone, Werte?!

\section{Systemmodell}

% TODO nur Anregungen was hier stehen soll!
\section{Produktleistung}
* Angaben über die mögliche Anzahl ausgeschlossener Nutzer
* Dauer des Groupkeyupdates, da in dieser Zeit nichts empfangen werden kann
* Machbarkeit der Keygeneration

\section{GUI}

\section{Testfälle}

\section{Entwicklungsumgebung}

\clearpage
\newglossaryentry{broadcastenc}
{
  name=Broadcast-Verschl"usselung,
  description={Ein Verschl"usselungsverfahren f"ur unidirektionale Streams, bei dem der
  Sender die Untermenge der Empf"anger bestimmen kann, die in der Lage ist, den Stream
  zu entschl"usseln}
}
\newglossaryentry{sessionkey}
{
  name=Session-Key,
  description={Der symmetrische Schl"ussel, mit dem bei der Broadcast"=Verschl"usselung die
  Nutzdaten verschl"usselt werden. Jeder nicht ausgeschlossene Client muss diesen Schl"ussel
  aus den Server-Nachrichten berechnen k"onnen}
}
\newglossaryentry{server}
{
  name=Server,
  description={Eine Instanz, der in einem Computersystem Daten oder Ressourcen zur Verfügung stellt}
}
\newglossaryentry{client}
{
  name=Client,
  description={Eine Instanz, die Daten oder Anwendungen von einem Server anfordert}
}
\newglossaryentry{traffic}
{
  name=Traffic,
  description={Durch Netzwerkübertragungen entstehender Datenfluss}
}
\newglossaryentry{key}
{
  name=Schlüssel,
  description={Ein Schlüssel in der Kryptologie ist zusätzliche Information, die man benötigt um eine
	Nachricht zu chiffrieren bzw. dechiffrieren. Normalerweise besteht ein Schlüssel aus einer Folge von
	Zahlen oder Buchstaben, die entweder nur dem Empf"anger, oder sowohl Absender als auch Empfänger
        einer Nachricht bekannt sind}
}
\newglossaryentry{hostname}
{
  name=Hostname,
  description={Die eindeutige Bezeichnung, mit der ein Rechner er im Netzwerk angesprochen wird}
}
\newglossaryentry{gui}
{
  name=GUI (Graphical User Interface),
  description={Eine Software-Komponente, die einem Computerbenutzer die Interaktion mit der Maschine
   über grafische Symbole erlaubt}
}

% Setze den richtigen Namen für das Glossar
\renewcommand*{\glossaryname}{\section{\glossarName}}

% Drucke das gesamte Glossar
\glsaddall
\printglossaries


\end{document}
