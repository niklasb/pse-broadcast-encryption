\begin{illustration}{Client/Server-Modell unserer Broadcasting-Anwendung}

\tikzset{
  rect/.style={draw,fill=green!15,minimum height=0.8cm,rectangle},
  box/.style={
    draw=blue!50!white,
    line width=1pt,
    dash pattern=on 1pt off 4pt on 6pt off 4pt,
    inner sep=4mm, rectangle, rounded corners
  },
}

\begin{tikzpicture}[auto,node distance=1.5cm]

\node[rect,minimum width=2cm](server) {\textbf{Server}};
\node[rect,minimum width=2cm,xshift=5cm,right of=server](view) {View};
\node[rect,minimum width=2cm,below of=view](controller) {Controller};
\node[box,fit=(view.north west) (view.north east)
              (controller.south east) (controller.south west),
      inner sep=0.3cm](client) {};

\node[rect,fill=red!15,below of=client,xshift=-3cm,yshift=-2cm](library)
    {\textbf{Broadcast-Library}};
\node at (client.north) [above, inner sep=2mm] {\textbf{Client}};

\draw [->] (server.east) -- (client.west |- server.east)
             node[pos=.5]{sendet an}
             node[pos=0.94]{*};

\path[->]
  (view) edge node {} (controller)
  (controller) edge node {} (view)
  (controller) edge node {benutzt} (library)
  (server) edge node[pos=.87,xshift=-1.8cm] {benutzt} (library)
  ;

\end{tikzpicture}

\end{illustration}
