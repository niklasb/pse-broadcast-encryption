\newglossaryentry{broadcastenc}
{
  name=Broadcast-Verschl"usselung,
  description={Ein Verschl"usselungsverfahren f"ur unidirektionale Streams, bei dem der
  Sender die Untermenge der Empf"anger bestimmen kann, die in der Lage ist, den Stream
  zu entschl"usseln}
}
\newglossaryentry{sessionkey}
{
  name=Session-Key,
  description={Der symmetrische Schl"ussel, mit dem die Nutzdaten verschl"usselt werden.
  Jeder nicht ausgeschlossene Client muss diesen Schl"ussel aus den Server-Nachrichten
  berechnen k"onnen}
}
\newglossaryentry{Client}
{
  name=Client,
  description={Ein Client (deutsch: Kunde) ist eine Software, die Daten oder Anwendungen 
  von einem Server anfordert.}
}
\newglossaryentry{Server}
{
  name=Server,
  description={Ein Server (deutsch: Diener) ist in der Informatik ein Dienstleister, 
	der in einem Computersystem Daten oder Ressourcen zur Verfügung stellt.}
}
\newglossaryentry{Payload}
{
  name=Payload,
  description={Payload (deutsch: Nutzdaten) sind während einer Kommunikation zwischen 
  zwei Partnern transportierten Daten eines Datenpakets, die keine Steuer- oder Protokollinformationen enthalten. 
  Nutzdaten sind unter anderem Sprache, Text, Zeichen, Bilder und Töne.}
}
\newglossaryentry{Traffic}
{
  name=Traffic,
  description={Traffic (deutsch: Daten-Verkehr) bezeichnet durch Abrufe von Webdokumenten 
  und anderen Dateien einer Website entstehende Datentransfer-Volumina zwischen einem Server und dem Client-Programm.}
}
\newglossaryentry{Schlüssel}
{
  name=Schlüssel,
  description={Ein Schlüssel in der Kryptologie ist zusätzliche Information die man benötigt um eine 
	Nachricht zu chiffrieren bzw. dechiffrieren. Normalerweise besteht ein Schlüssel aus einer Folge von 
	Zahlen oder Buchstaben die nur dem Absender und dem Empfänger einer Nachricht bekannt sind.}
}
\newglossaryentry{Hostname}
{
  name=Hostname,
  description={Der Hostname (auch Sitename) ist die eindeutige Bezeichnung eines Rechners in einem Netzwerk, 
	mit der er im Netzwerk angesprochen wird.}
}
\newglossaryentry{GUI}
{
  name=GUI,
  description={Graphical user interface(GUI) (deutsch : grafische Benutzeroberfläche) ist eine Software-Komponente, 
	die einem Computerbenutzer die Interaktion mit der Maschine über grafische Symbole erlaubt.}
}

% Setze den richtigen Namen für das Glossar
\renewcommand*{\glossaryname}{\section{\glossarName}}

% Drucke das gesamte Glossar
\glsaddall
\printglossaries

