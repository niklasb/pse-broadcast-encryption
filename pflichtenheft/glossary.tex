\newglossaryentry{broadcastenc}
{
  name=Broadcast-Verschl"usselung,
  description={Ein Verschl"usselungsverfahren f"ur unidirektionale Streams, bei dem der
  Sender die Untermenge der Empf"anger bestimmen kann, die in der Lage ist, den Stream
  zu entschl"usseln}
}
\newglossaryentry{sessionkey}
{
  name=Session-Key,
  description={Der symmetrische Schl"ussel, mit dem bei der Broadcast"=Verschl"usselung die
  Nutzdaten verschl"usselt werden. Jeder nicht ausgeschlossene Client muss diesen Schl"ussel
  aus den Server-Nachrichten berechnen k"onnen}
}
\newglossaryentry{server}
{
  name=Server,
  description={Eine Instanz, der in einem Computersystem Daten oder Ressourcen zur Verfügung stellt.}
}
\newglossaryentry{client}
{
  name=Client,
  description={Eine Instanz, die Daten oder Anwendungen von einem Server anfordert}
}
\newglossaryentry{traffic}
{
  name=Traffic,
  description={Durch Netzwerkübertragungen entstehender Datenfluss}
}
\newglossaryentry{key}
{
  name=Schlüssel,
  description={Ein Schlüssel in der Kryptologie ist zusätzliche Information, die man benötigt um eine
	Nachricht zu chiffrieren bzw. dechiffrieren. Normalerweise besteht ein Schlüssel aus einer Folge von
	Zahlen oder Buchstaben, die entweder nur dem Empf"anger, oder sowohl Absender als auch Empfänger
        einer Nachricht bekannt sind}
}
\newglossaryentry{hostname}
{
  name=Hostname,
  description={Die eindeutige Bezeichnung, mit der ein Rechner er im Netzwerk angesprochen wird}
}
\newglossaryentry{gui}
{
  name=GUI (Graphical User Interface),
  description={Eine Software-Komponente, die einem Computerbenutzer die Interaktion mit der Maschine
   über grafische Symbole erlaubt}
}

% Setze den richtigen Namen für das Glossar
\renewcommand*{\glossaryname}{\section{\glossarName}}

% Drucke das gesamte Glossar
\glsaddall
\printglossaries

