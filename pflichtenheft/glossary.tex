%%%%%%%%%%%%%%%%%%%%%%%%%%%%%%%%%%%%%%%%%%%%%%%%%%%%%%%%%%%%%%%%%%%%%%
% Begriffslexikon zur Beschreibung des Produkts						 %
%%%%%%%%%%%%%%%%%%%%%%%%%%%%%%%%%%%%%%%%%%%%%%%%%%%%%%%%%%%%%%%%%%%%%%
\newglossaryentry{uebungsleiter}
{
  name=Übungsleiter,
  description={Organisator des Übungsbetriebs, administrativer Benutzer innerhalb der Anwendung}
}
\newglossaryentry{student}
{
  name=Student,
  description={Teilnehmer der Übungsbetriebs, muss Übungsaufgaben bearbeiten und Lösungen für Übungsaufgaben abgeben (u.U. durch Dateiupload in der Anwendung). Ist genau einem Tutor zugeordnet}
}
\newglossaryentry{tutor}
{
  name=Tutor,
  description={Organisiert Tutorien für die ihm zugeordneten Studenten. Innerhalb der Anwendung hat er die Aufgabe, die Abgaben seiner Studenten mit Punkten zu bewerten}
}
\newglossaryentry{uebungsblatt}
{
  name=Übungsblatt,
  description={Eine Sammlung von Übungsaufgaben, die eine gemeinsame Abgabedeadline besitzen. Von den Übungsleitern erstellt und verwaltet}
}
\newglossaryentry{uebungsaufgabe}
{
  name=Übungsaufgabe,
  description={Eine konkrete Arbeitsanweisung oder Frage für die Studenten. Kann entweder in der Anwendung als elektronisch zu bearbeiten markiert sein, in welchem Fall die Abgabe einer Lösung in elektronischer Form erfolgen {\em muss}, oder wird andernfalls auf Papier bearbeitet}
}


% Setze den richtigen Namen für das Glossar
\renewcommand*{\glossaryname}{\section{\glossarName}}

% Drucke das gesamte Glossar
\glsaddall
\printglossaries

% Trage das Glossar in das Inhaltsverzeichnis ein
\stepcounter{section}
\addcontentsline{toc}{section}{\numberline {\thesection} \glossarName}
